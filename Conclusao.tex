

\documentclass[a4paper,12pt] {article}
\usepackage{times}
\usepackage{geometry}
\usepackage[english, brazil]{babel}
\usepackage[utf8]{inputenc}
\usepackage{indentfirst}
\usepackage[T1]{fontenc}
\usepackage{nomencl}
\usepackage{graphicx}
\graphicspath{{figuras/}}
\usepackage{amsmath, amssymb, amsthm}
\usepackage{longtable}
\usepackage{caption}
\usepackage{setspace}
\usepackage{booktabs}

\begin{document}
	
	\section{Conclusão}

Neste artigo avaliamos os transbordamentos de volatilidade entre o mercado do etanol e de commodities agrícolas no Brasil utilizando o Modelo Multivariado GARCH Baba–Engle–Kraft–Kroner (BEKK). São usados dados diários de janeiro de 2010 a dezembro de 2016 para commodities soja, etanol e açúcar. 

Os resultados sugerem que os preços do etanol, soja e açúcar estão relacionados por dinâmica de equilíbrio de longo prazo. Além disso, a soja impacta os preços do etanol, enquanto o contrário não é válido. Este resultado pode ser explicado pelo rápido aumento da produção de soja no Brasil nos últimos cinco anos. Para a volatilidade obtemos que  somente o soja sofre impactos significativos das outras commodities, enquanto o açúcar e o etanol sofrem influência apenas de suas próprias inovações e volatilidades. Também verificamos que  o etanol possui um efeito mais forte sobre o soja do que o açúcar, pois o etanol afeta o soja por meio da matriz de covariância condicional e incondicional, enquanto o açúcar impacta o soja apenas pela matriz de covariâncias condicional.

Nossos resultados têm implicações políticas importantes. Diferentemente de Serra (2011) e Cabrera e Shutz (2015) e como Gardebroek e Hernandez(2013) encontramos evidências significativas de transbordamento de volatilidades do mercado do etanol para os mercados de alimentos. Isso é consistente ao fato que as culturas de cana de açúcar e soja concorrem com mesmo espaço territorial no Brasil. Conclui-se que a preocupação do etanol como causa da instabilidade dos preços dos alimentos pode ser justificada pelos resultados encontrados. 

Para pesquisas futuras pretendemos usar o modelo Herwartz e Lütkepohl (2011)  para medir as inter-relações dos preços entre as commodities. O procedimento  acomoda a heterocedasticidade na estimação da relação de cointegração, o que pode levar  a  um ganho de eficiência nas estimações. Apesar de ter apresentado artigo, não conseguimos realizar as estimações por este método.

\end{document}

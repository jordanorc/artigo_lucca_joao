	\section{Introdução}


Na últimos dez anos ocorreu um grande aumento na produção e consumo de energia alternativa, sobretudo do etanol. A produção mundial do etanol quase duplicou nos últimos cinco anos. O Brasil se destaca como uns dos maiores produtores do  biocombustível, segundo Departamento de Pesquisas e Estudos Econômicos (DEPEC) cerca de 30\%  de tudo que é produzido no mundo. O aumento da produção do etanol levanta questões importantes como a sustentabilidade dos bicombustíveis e  segurança alimentar. Existe uma preocupação mundial sobre o aumento das culturas voltadas ao fato de que a produção de biocombustíveis possa causar  maiores níveis e  instabilidade de preços dos alimentos, principalmente devido a substituição de culturas alimentícias por  plantações direcionadas a produção de biocombustíveis, geralmente culturas forrageiras como a cana-de-açúcar.   As relações potenciais dos preços também podem ser exacerbadas ou enfraquecidas por políticas específicas de favorecimento, como subsídios.

A volatilidade de preços também traz custos adicionais para os participantes destes  mercados, tanto produtores como consumidores. Primeiramente, os preços mais  voláteis  elevam os custos para agricultores gerenciar os riscos de preços, o que pode alterar as decisões entre hedge e investimento \cite{wu_volatility_2011, gardebroek_energy_2013}. Como muitas \emph{commodities} estão inclusas no mercado financeiro, a volatilidade  também eleva o custo de gerenciamento do risco para os investidores deste setor.  Já os consumidores podem ter seus níveis de bem estar frequentemente alterados, devido à instabilidade de preços, modificando constantemente sua cesta de consumo. No nível macro, \citeonline{byrne_primary_2013} indicam que o aumento da volatilidade dos preços agrícolas também afeta o desenho e a efetividade das políticas de estabilização de preços. Em termos gerais, a instabilidade de preços dos alimentos não é apenas uma questão de segurança alimentar. 

 A preocupação sobre a volatilidade dos preços dos alimentos acentua-se ao observarmos que durante o período recente de  expansão dos biocombustíveis também ocorreu um rápido crescimento e aumento da instabilidade de preços  de algumas \emph{commodities} agrícolas. Porém, a priori não podemos afirmar as causas da instabilidade de preços sem uma investigação profunda. Algumas evidências empíricas sugerem que as preocupações com o biodiesel como causa dos altos e voláteis  preços dos alimentos são injustificadas  \cite{lopez_cabrera_volatility_2016} .
 
O recente artigo contribui sobre o assunto ao estudar as inter-relações  entre preços e transmissões de volatilidades do etanol e dos produtos agrícolas no Brasil usando dados diários de janeiro de 2010 a dezembro de 2016. Pretende-se verificar se existem transbordamentos e como eles se comportam ao longo tempo . A maioria dos estudos nesta área abordam as dependências entre os níveis dos preços. A literatura sobre transmissões de volatilidade ainda é escassa. Para tanto serão utilizadas três séries temporais de \emph{commodities}: etanol, açúcar e soja. 

A escolhas das \emph{commodities} não são ao acaso. A teoria econômica baseada em fundamentos de mercado e atividades de arbitragem sugere que os preços do etanol, soja e açúcar estão inter-relacionados \cite{de_gorter_welfare_2007}. O açúcar e etanol são relacionados, pois são produzidos a partir da mesma cultura, que é a cana-de-açúcar. Já a soja pode ser indiretamente  afetada pelo aumento da demanda do etanol, principalmente pela substituição de área plantada para fins alimentícios por culturas voltadas a produção do biocombustível.  O grão é umas das principais fontes de proteína vegetal, cada vez mais usada para alimentação humana.  Além disso, o Brasil é um dos grandes produtores da soja, aproximadamente 30\% da produção mundial. As culturas de soja e cana-de-açúcar  concorrem pelo mesmo espaço territorial no Brasil, principalmente nos estados de Mato Grosso, Mato Grosso do Sul, Goiás, Paraná e São Paulo. 

Existem outras contribuições no artigo. Além de usar dados diários, a maioria dos trabalhos usam dados semanais ou mensais, e investigar as inter-relações usando o produto soja, apenas um trabalho conhecido usou o grão, no artigo usamos uma nova metodologia desenvolvida  por \citeonline{herwartz_generalized_2011} , pouco usada na literatura, para medir as inter-relações dos preços entre as \emph{commodities}. O procedimento  acomoda a heterocedasticidade na estimação da relação de cointegração, o que pode significar um ganho de eficiência em relação aos demais métodos. Primeiramente,  analisamos a média condicional usando um modelo de correção de erros (VECM) e modelamos a  volatilidade através de um modelo multivariado que permite co-movimentos de segunda ordem entre as séries (MGARCH).  Todos parâmetros  são estimados conjuntamente por meio de um estimador de mínimo quadrados generalizados factível (FGLS).

O restante deste artigo está estruturado da seguinte forma. A próxima seção apresentamos uma revisão da literatura sobre as transmissões de preços e os efeitos de transbordamentos entre os mercados energético e agrícola . Na secção 3 descrevemos  a estratégia econométrica e na seção 4 apresentamos os resultados da análise empírica para dados reais. Por fim, concluímos na seção 6. Todas as estimações foram realizadas no software estatístico R e as séries de preços foram fornecidas pelo Centro de Estudos Avançados em Economia Aplicada (CEPEA).
\section{Conclusão e Comentários Finais}

Neste artigo avaliamos os transbordamentos de volatilidade entre o mercado do etanol e de commodities agrícolas no Brasil utilizando o Modelo Multivariado GARCH Baba–Engle–Kraft–Kroner (BEKK). São usados dados diários de janeiro de 2010 a dezembro de 2016 para commodities soja, etanol e açúcar. Inicialmente foi estimado um modelo de vetor de correção de erros para filtrar as séries de sua relação de longo prazo e depois modelar suas respectivas volatilidades sem a interferência do co-movimento entre as médias dos preços.

Os resultados sugerem que os preços do etanol, soja e açúcar estão relacionados por dinâmica de equilíbrio de longo prazo. Além disso, a soja é impactada pelo  preço do etanol, enquanto o contrário não é válido. Este resultado pode ser explicado pelo conflito agrário existente com a expansão da cultura de cana-de-açúcar. Para a volatilidade obtemos que  somente o soja sofre impactos significativos das outras commodities, enquanto o açúcar e o etanol sofrem influência apenas de suas próprias inovações e volatilidades. Também verificamos que  o etanol possui um efeito mais forte sobre o soja do que o açúcar, pois o etanol afeta o soja por meio da matriz de covariância condicional e incondicional (efeitos de longo prazo e curto prazo), enquanto o açúcar impacta o soja apenas pela matriz de covariâncias condicional (efeito de curto prazo).

Nossos resultados têm implicações políticas importantes. Diferentemente de \citeonline{serra_volatility_2011} e \citeonline{lopez_cabrera_volatility_2016} e como \citeonline{gardebroek_energy_2013} encontramos evidências significativas de transbordamento de volatilidades do mercado do etanol para os mercados de alimentos (soja). Isso é consistente ao fato que as culturas de cana de açúcar e soja concorrem com mesmo espaço territorial no Brasil. Conclui-se que a preocupação do etanol como causa da instabilidade dos preços dos alimentos pode ser justificada pelos resultados encontrados. 

Para pesquisas futuras pretendemos usar o modelo \citeonline{herwartz_generalized_2011}  para medir as inter-relações dos preços entre as commodities. O procedimento  acomoda a heterocedasticidade na estimação da relação de cointegração, o que pode levar  a  um ganho de eficiência nas estimações. 

Como comentários finais, apesar de ter apresentado o artigo, não conseguimos realizar as estimações pelo modelo proposto por  \citeonline{herwartz_generalized_2011}, então utilizamos o modelo VECM tradicional que não controla para a heterocedasticidade das séries. Outro modelo  a ser testado é o modelo de Dinâmico de Correlação Cruzada (DCC), que seria o recíproco do modelo BEKK. O modelo BEKK sofre com o problema de dimensionalidade, sendo que estima muitos parâmetros, e um modelo com mais de três variáveis se torna complexo \cite{tsay_multivariate_2013}. Entretanto, o modelo DCC supõe que  variáveis diferentes causem o mesmo efeito nas volatilidades, o que reduz consideravelmente o número de parâmetros a ser estimados, mas empiricamente este modelo tende a não apresentar resultados significativos \cite{tsay_multivariate_2013}.
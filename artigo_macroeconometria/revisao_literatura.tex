	\section{Revisão de Literatura}
			
	Estudos sobre a transmissão da  volatilidade entre os mercados de energia e agricultura ainda são escassos \cite{serra_volatility_2011, gardebroek_energy_2013, lopez_cabrera_volatility_2016}. A maioria dos estudos sobre inter-relações entre os biocombustíveis e os preços das \emph{\emph{commodities}} agrícolas são para  Brasil e os EUA,  maiores produtores de biocombustíveis. A seguir, apresentamos uma breve revisão  da literatura recente sobre as transmissões de preços relacionadas aos biocombustíveis.   
	
	\citeonline{balcombe_bayesian_2008} investigam  ajuste não-linear para o equilíbrio de longo prazo entre petróleo bruto, etanol e açúcar no Brasil utilizando técnicas bayesianas.  Encontram que os preços do petróleo determinam equilíbrio de longo prazo dos preços do açúcar e do etanol no Brasil. Também verificaram que mudanças nos preços do açúcar causaram mudanças nos preços do etanol, mas não o contrário, sugerindo uma hierarquia causal entre os mercados: do petróleo para açúcar e do açúcar para o etanol. Segundo os autores o crescimento de longo prazo esperado levará a uma expansão da demanda de energia, o que sustentará os preços do petróleo em níveis elevados e garantirá a continuidade das relações de preços entre os mercados de energia e alimentos. 
	
	\citeonline{zhang_ethanol_2009} usa um modelo de correções vetoriais (VECM) e um modelo de heterocedasticidade  condicional autorregressiva múltipla (MGARCH) para investigar as transmissões de volatilidade entre os mercados do etanol, milho, petróleo e soja nos Estados Unidos entre o período de março de 1987 a dezembro de 2007. Como resultado, não encontra uma relação de longo prazo entre os preços das \emph{commodities} agrícolas e dos combustíveis. Um choque nos mercados dos combustíveis pode impactar os preços das \emph{commodities} agrícolas, porém há baixa persistência deste choque nos preços dos produtos agrícolas no curto prazo.  Segundo o autor  mercados descentralizados e livremente operacionais atenuarão a persistência desses choques.
	
	\citeonline{wu_volatility_2011} utilizam um modelo MGARH para examinar o transbordamento das volatilidades entre os mercados de petróleo e milho nos Estados Unidos entre o período de janeiro de 1992 a junho de 2009. Além de encontrar evidências de transbordamento das volatilidades entre os mercados, os  resultados indicam que o mercado de milho tornou-se muito mais conectado ao  do petróleo após a introdução da Lei de Política Energética de 2005. Também verificam que os transbordamentos de volatilidade entre os mercados são maiores em períodos de alta taxa de consumo de gasolina-etanol. 
	
	
	\citeonline{serra_volatility_2011}  usa o método desenvolvido por \citeonline{seo_asymptotic_2007}, um estimador de máximo verossimilhança do vetor de cointegração que estima conjuntamente o ECM e 	processo MGARCH, para avaliar a transmissão da volatilidade de preços do etanol ao longo do tempo e entre os mercados de açúcar e petróleo. Como resultado encontra que um aumento do preço do petróleo leva à  um novo equilíbrio caracterizado por preços mais altos do etanol.  Também verificam que choques positivos no mercado do petróleo e do açúcar levam a um aumento da volatilidade nos preços do etanol.  Em outro trabalho para o Brasil, \citeonline{serra_volatility_2011} investiga as transmissões de volatilidades dos preços entre os mercados de petróleo, etanol e açúcar  usando um modelo GARCH semi paramétrico e dados semanais entre período de julho de 2000 a novembro de 2010 para o Brasil. Os resultados sugerem que os preços do etanol e petróleo, tanto quanto os preços do açúcar e do etanol, estão relacionados por uma paridade de equilíbrio de longo prazo. As relações demostram que o preço do etanol aumenta com o preço da açúcar e do petróleo. Para avaliar os efeitos de transbordamento da volatilidade a autora usou um modelo BEKK. Como resultado, obteve que choques do petróleo e no mercado de açúcar levam a uma maior volatilidade nos preços do etanol. No entanto, o mercado do etanol possui pouca capacidade de aumentar a instabilidade nos mercados de petróleo e açúcar. Por meio da comparação  das variâncias simuladas sob os modelos  GARCH paramétrico e semiparamétrico a autora conclui que os estimadores  semiparamétricos podem captar com mais precisão o comportamento dos preços durante os períodos turbulentos.
	
	\citeonline{serra_price_2013} usam dados mensais de janeiro de 1990 a dezembro de 2010 para estudar a volatilidade do preço do milho nos Estados Unidos. Para tanto os autores estimam um modelo MGARCH, que permite variáveis exógenas,  parametricamente e semiparametricamente. Como resultado, encontram evidências de transmissão de volatilidade entre os mercados do etanol e milho. Também verificam que estoques de milho diminuem a volatilidade dos preços deste produto, enquanto instabilidade econômica leva a um aumento da instabilidade dos preços. Os autores destacam a relevância de estender as análises de transbordamento de volatilidade entre os mercados de alimentos e energia considerando um conjunto mais amplo de variáveis explicativas, uma vez que  variáveis exógenas se mostraram importantes nos modelos de volatilidade.
	
	\citeonline{gardebroek_energy_2013} usam a abordagem MGARCH para examinar a transmissão de volatilidade nos preços do petróleo, etanol e milho entre os anos de 1997 e 2011 para os Estados Unidos.  Particularmente, os autores estimam o modelo T-BEKK e um modelo DCC. Os resultados indicam que não existem transbordamentos cruzados para os retornos médios entre os   mercados de petróleo, etanol e milho. O retorno médio observado de cada \emph{commodity} é influenciado somente pelo retorno defasado do mesmo mercado e não pelos retornos defasados nos outros mercados. As \emph{commodities} também apresentaram efeitos de volatilidade própria significavas e elevadas, sendo que o etanol mostrou uma mais baixa persistência da própria volatilidade. Os efeitos cruzados para volatilidade se apresentaram significativos, porém com menor magnitude do que a volatilidade própria. Para estudar as interações ao longo do tempo entre as \emph{commodities} os autores estimaram os modelos para subamostras. Entre os resultados, encontraram  que o etanol e milho apresentaram uma maior interação nos últimos anos da amostra, período no qual o etanol tornou-se o principal substituto para a gasolina. Além disso, as estimações mostraram importantes efeitos de transbordamento do milho para o etanol e não o contrário.  Os resultados não fornecem evidências que a volatilidade nos mercados de energia estimulam a volatilidade de preços no mercado do milho. 
	
	\citeonline{lopez_cabrera_volatility_2016} investigam as ligações de curto e longo prazo entre os preços de combustíveis e de \emph{commodities} agrícolas na Alemanha no período de 2003 a 2012. No artigo as autoras usam um novo estimador de mínimos quadrados generalizados  factível  introduzido por \citeonline{herwartz_generalized_2011}, que acomoda a heterocedasticidade no procedimento de estimação da relação de cointegração. Mais precisamente é calculado VECM para filtrar os comovimentos entre as séries. As tendências de curto prazo, resíduos do VECM, são usadas para modelar as inter-relações de volatilidade através de um modelo MGARCH. Todos os parâmetros, tanto do VECM e MGARCH, são calculados conjuntamente por meio de estimador FGLS. Como resultado encontram que as volatilidades e correlações são altamente persistentes no curto prazo. Verificam que os preços se movem juntos e preservam um equilíbrio de longo prazo,  onde os preços do biodiesel se ajustam aos preços do petróleo bruto e da colza. Também encontram que a volatilidade do biodiesel está apenas ligada fracamente à volatilidade do petróleo bruto e da colza tanto no curto como no longo prazo, enquanto a ligação entre a volatilidade da colza e do petróleo  está aumentando nos últimos anos. Com base nestes resultados, concluem que as preocupações com o biodiesel como causa dos altos e voláteis  preços dos alimentos são  injustificadas.
	
	Como podemos observar, apesar de existirem alguns trabalhos sobre as transmissões de preços e volatilidade entre os mercados de combustíveis e alimentos para o Brasil, não existem estudos sobre inter-relações do mercado de biocombustível com a \emph{commodity} soja. O trabalho busca compreender estas inter-relações, uma vez que a cana-de-açúcar (matéria prima para etanol e açúcar) e a soja estão entre os principais produtos agrícolas produzidos no Brasil. Além disso, a grande maioria dos estudos usaram dados mensais ou semanais, sendo que este trabalho utiliza dados diários. Trabalhos com esta periodicidade não foram encontrados para o mercado de \emph{commodities} brasileiro. Com isso questões de microestrutura do mercado de \emph{commodities} podem ser melhor captadas, sendo que a periodicidade tende a influenciar os resultados de estudos de volatilidade e conforme \citeonline{moreira_o_2004} uma menor frequência de dados melhoram o ajuste do modelo e sua previsibilidade.
	
	
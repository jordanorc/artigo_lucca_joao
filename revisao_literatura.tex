\documentclass[]{article}
\usepackage{lmodern}
\usepackage{amssymb,amsmath}
\usepackage{ifxetex,ifluatex}
\usepackage{fixltx2e} % provides \textsubscript
\ifnum 0\ifxetex 1\fi\ifluatex 1\fi=0 % if pdftex
  \usepackage[T1]{fontenc}
  \usepackage[utf8]{inputenc}
\else % if luatex or xelatex
  \ifxetex
    \usepackage{mathspec}
  \else
    \usepackage{fontspec}
  \fi
  \defaultfontfeatures{Ligatures=TeX,Scale=MatchLowercase}
\fi
% use upquote if available, for straight quotes in verbatim environments
\IfFileExists{upquote.sty}{\usepackage{upquote}}{}
% use microtype if available
\IfFileExists{microtype.sty}{%
\usepackage{microtype}
\UseMicrotypeSet[protrusion]{basicmath} % disable protrusion for tt fonts
}{}
\usepackage[margin=1in]{geometry}
\usepackage{hyperref}
\hypersetup{unicode=true,
            pdfborder={0 0 0},
            breaklinks=true}
\urlstyle{same}  % don't use monospace font for urls
\usepackage{longtable,booktabs}
\usepackage{graphicx,grffile}
\makeatletter
\def\maxwidth{\ifdim\Gin@nat@width>\linewidth\linewidth\else\Gin@nat@width\fi}
\def\maxheight{\ifdim\Gin@nat@height>\textheight\textheight\else\Gin@nat@height\fi}
\makeatother
% Scale images if necessary, so that they will not overflow the page
% margins by default, and it is still possible to overwrite the defaults
% using explicit options in \includegraphics[width, height, ...]{}
\setkeys{Gin}{width=\maxwidth,height=\maxheight,keepaspectratio}
\IfFileExists{parskip.sty}{%
\usepackage{parskip}
}{% else
\setlength{\parindent}{0pt}
\setlength{\parskip}{6pt plus 2pt minus 1pt}
}
\setlength{\emergencystretch}{3em}  % prevent overfull lines
\providecommand{\tightlist}{%
  \setlength{\itemsep}{0pt}\setlength{\parskip}{0pt}}
\setcounter{secnumdepth}{5}
% Redefines (sub)paragraphs to behave more like sections
\ifx\paragraph\undefined\else
\let\oldparagraph\paragraph
\renewcommand{\paragraph}[1]{\oldparagraph{#1}\mbox{}}
\fi
\ifx\subparagraph\undefined\else
\let\oldsubparagraph\subparagraph
\renewcommand{\subparagraph}[1]{\oldsubparagraph{#1}\mbox{}}
\fi

%%% Use protect on footnotes to avoid problems with footnotes in titles
\let\rmarkdownfootnote\footnote%
\def\footnote{\protect\rmarkdownfootnote}

%%% Change title format to be more compact
\usepackage{titling}

% Create subtitle command for use in maketitle
\newcommand{\subtitle}[1]{
  \posttitle{
    \begin{center}\large#1\end{center}
    }
}

\setlength{\droptitle}{-2em}
  \title{}
  \pretitle{\vspace{\droptitle}}
  \posttitle{}
  \author{}
  \preauthor{}\postauthor{}
  \date{}
  \predate{}\postdate{}

\setlength\parindent{24pt}
\usepackage[english, brazil]{babel}
\usepackage[utf8]{inputenc}
\usepackage{longtable}
\usepackage{booktabs}
\usepackage{indentfirst}

\begin{document}

\section{Revisão de Literatura: Estudos de volatilidade no mercado
financeiro
brasileiro}\label{revisao-de-literatura-estudos-de-volatilidade-no-mercado-financeiro-brasileiro}

Moreira e Lemgruber (2004) investigaram o uso de dados de alta
frequência na estimação da volatilidade diária e intradiária do IBOVESPA
e no cálculo do valor em risco (VaR). Para isso usaram os modelos GARCH
e EGARCH em conjunto com métodos determinísticos de filtragem de
sazonalidade para a previsão da volatilidade e do VaR intradiários. Os
autores compararam seus resultados com o método não paramétrico e no
cálculo do VaR diário, dois métodos foram usados. O primeiro utiliza o
desvio padrão amostral com janela móvel e o segundo usa alisamento
exponencial. No cálculo do VaR diário, os dois métodos usados baseados
em informações intradiárias apresentaram bom desempenho. Ao calcularem o
VaR intradiário, seus resultados mostraram que a filtragem do padrão
sazonal é indispensável à obtenção de resultados satisfatórios por meio
dos modelos GARCH e EGARCH. A série utilizada pelos autores é o índice
IBOVESPA, registrados a cada 15 minutos durante o período de 06/04/1998
a 19/07/2001 e utilizaram o retorno logarítmico para as estimações. Os
autores também concluíram que o filtro de sazonalidade do intervalo
intradiário é mais importante para melhorar os resultados das estimações
do que o filtro referente ao dia da semana.

Carvalho et al. (2006) estimaram a volatilidade diária dos cinco ativos
amis negociados na bolsa de valores de São Paulo (na época BOVESPA e
BM\&F eram separadas). Os autores utilizaram dados intradiários e
utilizaram o estmador de variância realizada. Os autores concluíram que
os retornos diários padronizados pela volatilidade realizada são
aproximadamente normais. Também concluíram que as log-volatilidades
também apresentam distribuições bem próximas da normal. Em contraposição
com a literatura corrente até então, não encontraram evidências de
memória de longo prazo na série de volatilidade e um modelo de memória
curta foi suficiente para os autores modelarem e preverem as séries
diárias de volatilidade. Os modelo usados pelos autores foram o retorno
logarítimico padrozinado, EWMA, GARCH, EGARCH e GJR-GARCH.

Ceretta et al. (2011) inestigaram como a especificação da distribuição
influencia a performandce da previsão da volatilidade em dados
intradiários do IBovespa, usando o modelo APARCH. As previsões dos
autores foram realizadas supondo seis distribuições dinstintas: normal,
normal assimétrica, \emph{t-student}, \emph{t-student} assimétrica,
generalizada e generalizada assimétrica. Os resultados obtidos mostraram
que o modelo com distribuição \emph{t-student} assimétrica foi o que
melhor se ajustou aos dados dentro da amostra, porém, na previsão fora
da amostra, o modelo com distribuição normal apresentou melhor
desempenho.

Ceretta et al. (2011) encontraram também que uma modelagem feita a
partir de um asérie longa pode incorporar efeitos atípicos no modelo,
viesando a previsão. Os autores apoantam como uma possível solução
realizar um ajuste do modelo utilizando uma série menor, com menos
efeitos esporádicos, em que, segundo eles, a previsão poderia ter um
comportamento amis aproximado ao habitual para a série, o que
minimizaria o efeito de eventuais variações acentudas. Outro fato
destacado pelos autores é que o modelo que melhor se ajusta, nem sempre
fornecerá amelhor previsão. Isto, conforme Ceretta et al. (2011) ,
ressalta a importância da comparação entre os modelos estudados fora da
amostra para encontrar o modelo que melhor prevê o comportamento futuro
da série. a partir disso os autores sugerirm que o contexto
macroeconômico poderia influenciar tanto o ajuste quanto a previsão de
uma série financeira.

Junior e Pereira (2013) usaram dados intradiários com intervalo de tempo
de 5, 15 e 30 minutos para as ações mais comercializadas do índice
BOVESPA. Seu artigo analisou dois modelos para estimação e previsão da
volatilidade realizada: o modelo autorregressivo heterogÊneo de
volatilidade realizada (HAR-RV) e o modelo de amostragem de dados mistos
(MIDAS-RV). Os autores compararam previsões dentro e fora da amostra e
encontraram melhores resultados com o modelo MIDAS-RV para previsões
dentro da amostra. Para previsões fora da amostra não encontraram
diferença estatísticamente significativa entre os modelos. Por fim os
autores acharam evidência de que o uso de volatilidade realizada induz a
distribuições dos retornos padronizados próximas da normal.

Santos e Ziegelmann (2014) compararam diversos tipos de medidas de
volatilidade e seus modelos de previsão, das famílias de modelos
MIDAS-RV e HAR-RV. Realizaram comarações em temos da acurácia de
previsão da volatilidade fora da amostra e usaram também uma combinação
dos dois modelos. Para isso captaram dados intradiários do índice
IBOVESPA e calcularam medidas de volatilidade como variância realizada,
variação potência realizada e variação bi-potência realziada para serem
usados como regressores em ambos os modelos. Para a estimação usaram um
procedimento não paramétrico para mensurar separadamente a variação da
trajetória contínua da amostra e a parte de salto descontínuodo processo
de variação quadrático.Seus resultados em termos de erro quadrático
médio sigerem que regressores envolvendo medidas de volatilidade que são
robustos a saltos são melhores para previsão de volatilidade futura.
Encontraram ainda que previsõe sbaseadas nestes regressre não são
estatísticamente diferentes daqueles baseados em variância realizada.
Por fim, acharam que a performance de previsão das três abordagens são
estatísticamente equivalentes.

Vicente et al. (2014) examinaram se investidores possuem percepções
diferentes sobre a volatilidade diária de um ativo. Para isso, definiram
a volatilidade percebida pelo investidor como a distribuição dos
desvios-padrão dos retornos diários calculados de preços intradiários
coletados aleatoriamente. Os autores encontraram que esta distribuição
tem uma alto grau de dispersão, o que significa que diferentes
investidores podem não compartilhar a emsma opinião a respeito da
variabilidade do mesmo ativo. Entretanto, segundo Vicente et al. (2014)
a volatilidade de preços de fechamento é geralente menor que a mediana
da volatilidade percebida pelo investidor, enquanto a volatilidade de
preço de abertura é maior. Seus resultados indicaram que as
volatilidades usando amostras diárias tradicionais de retornos diários
podem não ser bons insumos de modelos financeiros, já que, conforme os
autores, eles podem não capturar adequadamente o risco em que os
investidores são expostos.

Ziegelmann, Borges e Caldeira (2014) exploraram diferentes estimadores
de matrix de covariância, tanto a condicional quanto a incondicional,
obtidas por dados intradiários. Tais medidas foram usadas para obter um
portfólio de variância mínima. Os dados foram coletados de forma
sincronizada e não sincronizada. Para fins de comparação os autores
também usaram dados diários. Em seu trabalho tambpem avaliaram as
performances fora da amostra dos índices obtidos de um portfólio de 30
ações comercializadas na BMF\&BOVESPA. Seus resultados mostraram que o
estimador da matrix de variância condicional dos retornos usando o
modelo escalar vt-VECH baseado em dados de alta frequÊncia leva a
melhoras substanciais de estimação, reduzindo o risco de portfólio,
aumentando o retorno médio ajustado pelo risco e reduzindo o
\emph{turnover} financeiro.

\begin{longtable}[]{@{}llll@{}}
\caption{Resumo dos estudos de volatilidade no mercado financeiro
brasileiro.}\tabularnewline
\toprule
\begin{minipage}[b]{0.17\columnwidth}\raggedright\strut
Autores\strut
\end{minipage} & \begin{minipage}[b]{0.17\columnwidth}\raggedright\strut
Objeto\strut
\end{minipage} & \begin{minipage}[b]{0.15\columnwidth}\raggedright\strut
Periodicidade\strut
\end{minipage} & \begin{minipage}[b]{0.23\columnwidth}\raggedright\strut
Método\strut
\end{minipage}\tabularnewline
\midrule
\endfirsthead
\toprule
\begin{minipage}[b]{0.17\columnwidth}\raggedright\strut
Autores\strut
\end{minipage} & \begin{minipage}[b]{0.17\columnwidth}\raggedright\strut
Objeto\strut
\end{minipage} & \begin{minipage}[b]{0.15\columnwidth}\raggedright\strut
Periodicidade\strut
\end{minipage} & \begin{minipage}[b]{0.23\columnwidth}\raggedright\strut
Método\strut
\end{minipage}\tabularnewline
\midrule
\endhead
\begin{minipage}[t]{0.17\columnwidth}\raggedright\strut
Moreira e Lemgruber (2004)\strut
\end{minipage} & \begin{minipage}[t]{0.17\columnwidth}\raggedright\strut
IBOVESPA\strut
\end{minipage} & \begin{minipage}[t]{0.15\columnwidth}\raggedright\strut
15 min e 1 dia\strut
\end{minipage} & \begin{minipage}[t]{0.23\columnwidth}\raggedright\strut
GARCH, EGARCH\strut
\end{minipage}\tabularnewline
\begin{minipage}[t]{0.17\columnwidth}\raggedright\strut
Carvalho et al. (2006)\strut
\end{minipage} & \begin{minipage}[t]{0.17\columnwidth}\raggedright\strut
Top 5 empresas IBOVESPA\strut
\end{minipage} & \begin{minipage}[t]{0.15\columnwidth}\raggedright\strut
15 min\strut
\end{minipage} & \begin{minipage}[t]{0.23\columnwidth}\raggedright\strut
EWMA, GARCH, EGARCH, GJR-GARCH, log retorno padronizado\strut
\end{minipage}\tabularnewline
\begin{minipage}[t]{0.17\columnwidth}\raggedright\strut
Ceretta et al. (2011)\strut
\end{minipage} & \begin{minipage}[t]{0.17\columnwidth}\raggedright\strut
IBOVESPA\strut
\end{minipage} & \begin{minipage}[t]{0.15\columnwidth}\raggedright\strut
15 min\strut
\end{minipage} & \begin{minipage}[t]{0.23\columnwidth}\raggedright\strut
APARCH\strut
\end{minipage}\tabularnewline
\begin{minipage}[t]{0.17\columnwidth}\raggedright\strut
Junior e Pereira (2013)\strut
\end{minipage} & \begin{minipage}[t]{0.17\columnwidth}\raggedright\strut
Top 5 empresas IBOVESPA\strut
\end{minipage} & \begin{minipage}[t]{0.15\columnwidth}\raggedright\strut
5, 15 e 30 min\strut
\end{minipage} & \begin{minipage}[t]{0.23\columnwidth}\raggedright\strut
HAR-RV e MIDAS-RV\strut
\end{minipage}\tabularnewline
\begin{minipage}[t]{0.17\columnwidth}\raggedright\strut
Morais et al. (2014)\strut
\end{minipage} & \begin{minipage}[t]{0.17\columnwidth}\raggedright\strut
Top 2 BMF\&BOVESPA\strut
\end{minipage} & \begin{minipage}[t]{0.15\columnwidth}\raggedright\strut
5 min\strut
\end{minipage} & \begin{minipage}[t]{0.23\columnwidth}\raggedright\strut
GARCH, EGARCH, CGARCH e TGARCH\strut
\end{minipage}\tabularnewline
\begin{minipage}[t]{0.17\columnwidth}\raggedright\strut
Santos e Ziegelmann (2014)\strut
\end{minipage} & \begin{minipage}[t]{0.17\columnwidth}\raggedright\strut
IBOVESPA\strut
\end{minipage} & \begin{minipage}[t]{0.15\columnwidth}\raggedright\strut
15 min\strut
\end{minipage} & \begin{minipage}[t]{0.23\columnwidth}\raggedright\strut
HAR, MIDAS e combinação HAR-MIDAS\strut
\end{minipage}\tabularnewline
\begin{minipage}[t]{0.17\columnwidth}\raggedright\strut
Vicente et al. (2014)\strut
\end{minipage} & \begin{minipage}[t]{0.17\columnwidth}\raggedright\strut
84 empresas da BMF\&BOVESPA\strut
\end{minipage} & \begin{minipage}[t]{0.15\columnwidth}\raggedright\strut
Amostragem aleatória de preços em um intervalo\strut
\end{minipage} & \begin{minipage}[t]{0.23\columnwidth}\raggedright\strut
Análise exploratória da Volatilidade Realizada\strut
\end{minipage}\tabularnewline
\begin{minipage}[t]{0.17\columnwidth}\raggedright\strut
Ziegelmann, Borges e Caldeira (2014)\strut
\end{minipage} & \begin{minipage}[t]{0.17\columnwidth}\raggedright\strut
Índice de 30 ações BMF\&BOVESPA\strut
\end{minipage} & \begin{minipage}[t]{0.15\columnwidth}\raggedright\strut
5 a 120 min\strut
\end{minipage} & \begin{minipage}[t]{0.23\columnwidth}\raggedright\strut
Covariância realizada, vt\_VECH escalar e MRK\strut
\end{minipage}\tabularnewline
\bottomrule
\end{longtable}

Fonte: Elaboração dos autores

\pagebreak

\section*{Referências}\label{referencias}
\addcontentsline{toc}{section}{Referências}

\setlength{\parindent}{0in}

\hypertarget{refs}{}
\hypertarget{ref-carvalho_modeling_2006}{}
CARVALHO, M. C.; FREIRE, M. A. S.; MEDEIROS, M. C.; SOUZA, L. R.
Modeling and Forecasting the Volatility of Brazilian Asset Returns: a
Realized Variance Approach. \textbf{Brazilian Review of Finance}, v. 4,
n. 1, p. 55--77, 2006. Disponível em:
\textless{}\url{http://bibliotecadigital.fgv.br/ojs/index.php/rbfin/article/view/1155}\textgreater{}.
Acesso em: 12/12/2016.

\hypertarget{ref-ceretta_intraday_2011}{}
CERETTA, P. S.; BARBA, F. G. DE; VIEIRA, K. M.; CASARIN, F. Intraday
volatility forecasting: analysis of alternative distributions.
\textbf{Brazilian Review of Finance}, v. 9, n. 2, p. 209--226, 2011.
Disponível em:
\textless{}\url{http://bibliotecadigital.fgv.br/ojs/index.php/rbfin/article/view/2586}\textgreater{}.
Acesso em: 12/12/2016.

\hypertarget{ref-junior_modeling_2013}{}
JUNIOR, M. V. W.; PEREIRA, P. L. V. Modeling and Forecasting of Realized
Volatility: Evidence from Brazil. \textbf{Brazilian Review of
Econometrics}, v. 31, n. 2, p. 315--337, 2013. Disponível em:
\textless{}\url{http://bibliotecadigital.fgv.br/ojs/index.php/bre/article/view/4056}\textgreater{}.
Acesso em: 12/12/2016.

\hypertarget{ref-morais_estimating_2014}{}
MORAIS, O.; VAL, F. DE F.; PINTO, A. C. F.; KL, M. C.; MACELLY.
Estimating the Volatility of Brazilian Equities using Garch-Type Models
and High-Frequency Volatility Measures. \textbf{Global Journal of
Management And Business Research}, v. 14, n. 5, 2014. Disponível em:
\textless{}\url{http://journalofbusiness.org/index.php/GJMBR/article/view/1520}\textgreater{}.
Acesso em: 12/12/2016.

\hypertarget{ref-moreira_o_2004}{}
MOREIRA, J. M. DE S.; LEMGRUBER, E. F. O uso de dados de alta freqüência
na estimação da volatilidade e do valor em risco para o IBOVESPA.
\textbf{Revista Brasileira de Economia}, v. 58, n. 1, p. 100--120, 2004.
Disponível em:
\textless{}\url{http://www.scielo.br/scielo.php?script=sci_abstract\&pid=S0034-71402004000100005\&lng=en\&nrm=iso\&tlng=pt}\textgreater{}.
Acesso em: 12/12/2016.

\hypertarget{ref-santos_volatility_2014}{}
SANTOS, D. G.; ZIEGELMANN, F. A. Volatility Forecasting via MIDAS, HAR
and their Combination: An Empirical Comparative Study for IBOVESPA.
\textbf{Journal of Forecasting}, v. 33, n. 4, p. 284--299, 2014.
Disponível em:
\textless{}\url{http://onlinelibrary.wiley.com/doi/10.1002/for.2287/abstract}\textgreater{}.
Acesso em: 12/12/2016.

\hypertarget{ref-vicente_assessing_2014}{}
VICENTE, J. V. M.; ARAUJO, G. S.; CASTRO, P. B. F. DE; TAVARES, F. N.
Assessing Day-to-Day Volatility: Does the Trading Time Matter?
\textbf{Brazilian Review of Finance}, v. 12, n. 1, p. 41--66, 2014.
Disponível em:
\textless{}\url{http://bibliotecadigital.fgv.br/ojs/index.php/rbfin/article/view/13483}\textgreater{}.
Acesso em: 12/12/2016.

\hypertarget{ref-ziegelmann_selection_2014}{}
ZIEGELMANN, F. A.; BORGES, B.; CALDEIRA, J. F. Selection of Minimum
Variance Portfolio Using Intraday Data: An Empirical Comparison Among
Different Realized Measures for BM\&FBovespa Data. \textbf{Brazilian
Review of Econometrics}, v. 35, n. 1, p. 23--46, 2014. Disponível em:
\textless{}\url{http://bibliotecadigital.fgv.br/ojs/index.php/bre/article/view/21453}\textgreater{}.
Acesso em: 12/12/2016.


\end{document}

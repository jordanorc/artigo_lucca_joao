\documentclass[]{article}
\usepackage{lmodern}
\usepackage{amssymb,amsmath}
\usepackage{ifxetex,ifluatex}
\usepackage{fixltx2e} % provides \textsubscript
\ifnum 0\ifxetex 1\fi\ifluatex 1\fi=0 % if pdftex
  \usepackage[T1]{fontenc}
  \usepackage[utf8]{inputenc}
\else % if luatex or xelatex
  \ifxetex
    \usepackage{mathspec}
  \else
    \usepackage{fontspec}
  \fi
  \defaultfontfeatures{Ligatures=TeX,Scale=MatchLowercase}
\fi
% use upquote if available, for straight quotes in verbatim environments
\IfFileExists{upquote.sty}{\usepackage{upquote}}{}
% use microtype if available
\IfFileExists{microtype.sty}{%
\usepackage{microtype}
\UseMicrotypeSet[protrusion]{basicmath} % disable protrusion for tt fonts
}{}
\usepackage[margin=1in]{geometry}
\usepackage{hyperref}
\hypersetup{unicode=true,
            pdfborder={0 0 0},
            breaklinks=true}
\urlstyle{same}  % don't use monospace font for urls
\usepackage{longtable,booktabs}
\usepackage{graphicx,grffile}
\makeatletter
\def\maxwidth{\ifdim\Gin@nat@width>\linewidth\linewidth\else\Gin@nat@width\fi}
\def\maxheight{\ifdim\Gin@nat@height>\textheight\textheight\else\Gin@nat@height\fi}
\makeatother
% Scale images if necessary, so that they will not overflow the page
% margins by default, and it is still possible to overwrite the defaults
% using explicit options in \includegraphics[width, height, ...]{}
\setkeys{Gin}{width=\maxwidth,height=\maxheight,keepaspectratio}
\IfFileExists{parskip.sty}{%
\usepackage{parskip}
}{% else
\setlength{\parindent}{0pt}
\setlength{\parskip}{6pt plus 2pt minus 1pt}
}
\setlength{\emergencystretch}{3em}  % prevent overfull lines
\providecommand{\tightlist}{%
  \setlength{\itemsep}{0pt}\setlength{\parskip}{0pt}}
\setcounter{secnumdepth}{5}
% Redefines (sub)paragraphs to behave more like sections
\ifx\paragraph\undefined\else
\let\oldparagraph\paragraph
\renewcommand{\paragraph}[1]{\oldparagraph{#1}\mbox{}}
\fi
\ifx\subparagraph\undefined\else
\let\oldsubparagraph\subparagraph
\renewcommand{\subparagraph}[1]{\oldsubparagraph{#1}\mbox{}}
\fi

%%% Use protect on footnotes to avoid problems with footnotes in titles
\let\rmarkdownfootnote\footnote%
\def\footnote{\protect\rmarkdownfootnote}

%%% Change title format to be more compact
\usepackage{titling}

% Create subtitle command for use in maketitle
\newcommand{\subtitle}[1]{
  \posttitle{
    \begin{center}\large#1\end{center}
    }
}

\setlength{\droptitle}{-2em}
  \title{}
  \pretitle{\vspace{\droptitle}}
  \posttitle{}
  \author{}
  \preauthor{}\postauthor{}
  \date{}
  \predate{}\postdate{}

\setlength\parindent{24pt}
\usepackage[english, brazil]{babel}
\usepackage[utf8]{inputenc}
\usepackage{longtable}
\usepackage{booktabs}
\usepackage{indentfirst}

\begin{document}

\section{Revisão de Literatura: Estudos de volatilidade no mercado
financeiro
brasileiro}\label{revisao-de-literatura-estudos-de-volatilidade-no-mercado-financeiro-brasileiro}

Pesquisas usando dados de alta frequência ou dados com informações
financeiras intradiárias vêm sendo feitas no Brasil, principalmente
usando informações sobre o mercado de ações. Com este tipo de dado foram
realizados diversos estudos, em grande parte que estudam a volatilidade
do retorno nos ativos financeiros. No caso de mercado de ações os
estudos tratam sobre o melhor modelo de ajuste e acurácia de previsão.
Já o mercado de futuros, mais especifivamente o de \emph{commodities},
os estudos têm sido feitos para entender as características de
volatilidade do preço dos ativos, mas com dados de peridiocidade
predominantemente mensais ou semanais.

Moreira e Lemgruber (2004) investigaram o uso de dados de alta
frequência na estimação da volatilidade diária e intradiária do IBOVESPA
e no cálculo do valor em risco (VaR). Para isso usaram os modelos GARCH
e EGARCH em conjunto com métodos determinísticos de filtragem de
sazonalidade para a previsão da volatilidade e do VaR intradiários. Os
autores compararam seus resultados com o método não paramétrico e no
cálculo do VaR diário, dois métodos foram usados. O primeiro utiliza o
desvio padrão amostral com janela móvel e o segundo usa alisamento
exponencial. No cálculo do VaR diário, os dois métodos usados baseados
em informações intradiárias apresentaram bom desempenho. Ao calcularem o
VaR intradiário, seus resultados mostraram que a filtragem do padrão
sazonal é indispensável à obtenção de resultados satisfatórios por meio
dos modelos GARCH e EGARCH. A série utilizada pelos autores é o índice
IBOVESPA, registrados a cada 15 minutos durante o período de 06/04/1998
a 19/07/2001 e utilizaram o retorno logarítmico para as estimações. Os
autores também concluíram que o filtro de sazonalidade do intervalo
intradiário é mais importante para melhorar os resultados das estimações
do que o filtro referente ao dia da semana.

Carvalho et al. (2006) estimaram a volatilidade diária dos cinco ativos
mais negociados na bolsa de valores de São Paulo. Os autores utilizaram
dados intradiários e utilizaram o estimador de variância realizada. Os
autores concluíram que os retornos diários padronizados pela
volatilidade realizada são aproximadamente normais. Também concluíram
que as log-volatilidades também apresentam distribuições bem próximas da
normal. Em contraposição com a literatura corrente até então, não
encontraram evidências de memória de longo prazo na série de
volatilidade e um modelo de memória curta foi suficiente para os autores
modelarem e preverem as séries diárias de volatilidade. Os modelo usados
pelos autores foram o retorno logarítimico padrozinado, EWMA, GARCH,
EGARCH e GJR-GARCH.

Ceretta et al. (2011) inestigaram como a especificação da distribuição
influencia a performance da previsão da volatilidade em dados
intradiários do IBovespa, usando o modelo APARCH. As previsões dos
autores foram realizadas supondo seis distribuições dinstintas: normal,
normal assimétrica, \emph{t-student}, \emph{t-student} assimétrica,
generalizada e generalizada assimétrica. Os resultados obtidos mostraram
que o modelo com distribuição \emph{t-student} assimétrica foi o que
melhor se ajustou aos dados dentro da amostra, porém, na previsão fora
da amostra, o modelo com distribuição normal apresentou melhor
desempenho. Ceretta et al. (2011) encontraram também que uma modelagem
feita a partir de uma série longa pode incorporar efeitos atípicos no
modelo, viesando a previsão. Os autores apontam como uma possível
solução realizar um ajuste do modelo utilizando uma série menor, com
menos efeitos esporádicos, em que, segundo eles, a previsão poderia ter
um comportamento mais aproximado ao habitual para a série, o que
minimizaria o efeito de eventuais variações acentudas. Outro fato
destacado pelos autores é que o modelo que melhor se ajusta, nem sempre
fornecerá a melhor previsão. Isto, conforme Ceretta et al. (2011) ,
ressalta a importância da comparação entre os modelos estudados fora da
amostra para encontrar o modelo que melhor prevê o comportamento futuro
da série. A partir disso os autores sugerirm que o contexto
macroeconômico poderia influenciar tanto o ajuste quanto a previsão de
uma série financeira.

Junior e Pereira (2013) usaram dados intradiários com intervalo de tempo
de 5, 15 e 30 minutos para as ações mais comercializadas do índice
BOVESPA. Seu artigo analisou dois modelos para estimação e previsão da
volatilidade realizada: o modelo autorregressivo heterogêneo de
volatilidade realizada (HAR-RV) e o modelo de amostragem de dados mistos
(MIDAS-RV). Os autores compararam previsões dentro e fora da amostra e
encontraram melhores resultados com o modelo MIDAS-RV para previsões
dentro da amostra. Para previsões fora da amostra não encontraram
diferença estatísticamente significativa entre os modelos. Por fim os
autores acharam evidência de que o uso de volatilidade realizada induz a
distribuições dos retornos padronizados próximas da normal.

Santos e Ziegelmann (2014) compararam diversos tipos de medidas de
volatilidade e seus modelos de previsão, das famílias de modelos
MIDAS-RV e HAR-RV. Realizaram comparações em termos da acurácia de
previsão da volatilidade fora da amostra e usaram também uma combinação
dos dois modelos. Para isso captaram dados intradiários do índice
IBOVESPA e calcularam medidas de volatilidade como variância realizada,
variação potência realizada e variação bi-potência realziada para serem
usados como regressores em ambos os modelos. Para a estimação usaram um
procedimento não paramétrico para mensurar separadamente a variação da
trajetória contínua da amostra e a parte de salto descontínuodo de
processo de variação quadrático. Seus resultados em termos de erro
quadrático médio sugerem que regressores envolvendo medidas de
volatilidade que são robustos a saltos são melhores para previsão de
volatilidade futura. Encontraram ainda que previsões baseadas nestes
regressores não são estatísticamente diferentes daqueles baseados em
variância realizada. Por fim, acharam que a performance de previsão das
três abordagens são estatísticamente equivalentes.

Vicente et al. (2014) examinaram se investidores possuem percepções
diferentes sobre a volatilidade diária de um ativo. Para isso, definiram
a volatilidade percebida pelo investidor como a distribuição dos
desvios-padrão dos retornos diários calculados de preços intradiários
coletados aleatoriamente. Os autores encontraram que esta distribuição
tem uma alto grau de dispersão, o que significa que diferentes
investidores podem não compartilhar a mesma opinião a respeito da
variabilidade do mesmo ativo. Entretanto, segundo Vicente et al. (2014)
a volatilidade de preços de fechamento é geralente menor que a mediana
da volatilidade percebida pelo investidor, enquanto a volatilidade de
preço de abertura é maior. Seus resultados indicaram que as
volatilidades usando amostras diárias tradicionais de retornos diários
podem não ser bons insumos de modelos financeiros, já que, conforme os
autores, eles podem não capturar adequadamente o risco em que os
investidores são expostos.

Ziegelmann, Borges e Caldeira (2014) exploraram diferentes estimadores
de matriz de covariância, tanto a condicional quanto a incondicional,
obtidas por dados intradiários. Tais medidas foram usadas para obter um
portfólio de variância mínima. Os dados foram coletados de forma
sincronizada e não sincronizada. Para fins de comparação os autores
também usaram dados diários. Em seu trabalho também avaliaram as
performances fora da amostra dos índices obtidos de um portfólio de 30
ações comercializadas na BMF\&BOVESPA. Seus resultados mostraram que o
estimador da matriz de variância condicional dos retornos usando o
modelo escalar vt-VECH baseado em dados de alta frequência leva a
melhoras substanciais de estimação, reduzindo o risco de portfólio,
aumentando o retorno médio ajustado pelo risco e reduzindo o
\emph{turnover} financeiro.

Araujo e Montini (2015) proporam uma combinação da estimação de
volatilidade dos modelos HAR-RV e MIDAS-RV para responder a seguinte
questão: o tomador de decisão deveria selecionar a melhor projeção ou
projetar o futuro por meio da combinação de múltiplas projeções? Seus
resultados mostraram que o modelo HAR-RV apresentou melhor performance
para a amostra de dados utilizada. Ao comparar as projeções individuais
e métodos de combinação, a combinação de pesos iguais apresentou melhor
performance.

Percebe-se na literatura consultada, que os estudos que têm como onjeto
o mercado de ações são em sua maioria bastante técnicos e preocupados em
testar diversos métodos e modelos de estimação. A principal preocupação
deste tipo de pesquisa é obter o mehor ajuste de modelo e a melhor
previsão possível, usando os diversos métosos dispostos na literatura
para comparar seus restultados. Outra questão de destque é que na
literatura empírica sobre mercado de ações, o discussão dobre a
peridiocidade dos dados, principalmente de dados intradiários é bastante
difundida e objeto recorrente de análise. Já os estudos sobre preços e
retorno de commodities possuem uma característica um pouco diferente.

\subsection{\texorpdfstring{Estudos empíricos sobre volatilidade para
\emph{commodities}.}{Estudos empíricos sobre volatilidade para commodities.}}\label{estudos-empiricos-sobre-volatilidade-para-commodities.}

Devido à forma de obtnção dos dados, os estudos que tratam a
volatilidade doretorno de \emph{commodities} feitos no Brasil usam dados
geralmente semanais ou mensais. Estes estudos possuem preocupação
principal em estudar a persistência de choques, assimetria da
volatilidade eo efeito alavancagem. Como método, nos estudos que têm
como objeto retornos de \emph{commodities} é comum encontrar modelos da
família ARCH, sendo que não foram encontrados outros tipos de modelos
como HAR-RV e MIDAS-RV na literatura consultada.

Silva et al. (2005) examinaram o processo de volatilidade dos retornos
do café e da soja no Brasil usando dados mensais. Seus resultados
segurem fortes sinais de persistência e assimetria na volatilidade de
ambas as séries. Além disso, afirmam que seus resultados indicam que a
implementação de políticas que criem e facilitem o acesso e estimulem a
utilização de instrumentos de \emph{hedging} baseados no mercado podem
ser estratégias adequadas para tais setores lidarem com a persistência
de choques e volatilidade pronunciadas para os retornos destas
\emph{commodities}.

Lima, Góis e Ulises (2007) modelaram a previsão com diferenciação
inteira e fracionária, utilizando dados de preços futuros de
\emph{commodities} agrícolas. Os autores comparam as estimações de
modelos ARMA e ARIMA com os resultados obtidos pelo modelo ARFIMA. Para
avaliar o poder de previsão, os autores usaram o critério de erro
quadrado médio e também estimaram o termo de diferenciação \emph{d} para
examinar as características de longo prazo das séries. Seus resultados
indicaram que todas as séries de retornos de preços futuros são
estacionárias. Encontraram ainda que os modelos ARFIMA mostraram melhor
poder de previsão.

Silva (2008) analisou a volatilidade do retorno dos preços de boi gordo
no Estado de São Paulo. O autor examinou a persistência dos choques e
assimetrias na sua volatilidade usando os modelos ARCH e GARCH. Seus
resultados mostraram reações de persistência e assimetria na
volatilidade em que choques negativos e positivos têm impactos
diferentes sobre a volatilidade dos retornos dos preços do boi gordo.
Seus resultados foram corroborados pelos modelos EGARCH e TGARCH.

Teixeira et al. (2008) usou a família de modelos ARCH para analisar o
comportamento do retorno do cacau, do boi gordo e do café. Seus
resultados indicaram fortes sinais de persistÊncia para as três
\emph{commodities} e que os choques levam um longo tempo para
dissipar-se. Os autores ainda constataram que choques positivos e
negativos têm efeitos diferentes sobre a volatilidade, então os mercados
são assimétricos, e que o boi gordo e o cacau sofrem o efeito
alavancagem.

Pereira et al. (2010) analisaram os retornos da soja, café e boi gordo.
Para isso usaram modelos ARCH e calcularam o \emph{Value-at-Risk (VaR)}.
Seus resultados indicaram que a variabilidade das três
\emph{commodities} possui dependência condicional e que exite grande
persistência na resposta aos choques na variância. Observaram também que
os retornos do café e soja tiveram assimetrias nos choques positivos e
negativos, embora não encontraram o efeito alavancagem. As medidas dos
\emph{VaR} encontradas pelos autores mostraram maior potencial de perda
para os produtores de café, seguidos pelos da soja e de boi gordo.

Bodra (2012) usou um modelo de volatilidade estocástica com saltos para
estudar a dinâmica de preços do milho e da soja. Seus resultados
mostraram que um modelo de volatilidade estocástica pode ser bem
ajustado ao mercado de \emph{commodities} agrícolas e que o processo de
\emph{jump dffusion} pode representar bem os saltos deste mercado. Os
autores usaram um modelo de Monte Carlo de mínimos quadrados (LSM) para
a precificação das opções que foram utilizadas para formular uma
estratégia de \emph{hedge} de uma posição física de milho e de soja,
sendo que a eficiência dessa estratégia foi comparada com estratégias já
disponíveis no mercado.

Freitas et al. (2015) analisaram a persistência, a alavancagem e a
variância condicional dos retornos de commodities agropecuárias, usando
o modelo APARCH. As séries estudadas pelos autores foram, o açúcar, a
soja, o milho, o café, o algodão, o arroz, o trigo, o frango, o boi
gordo e o bezerro. Os autores encontraram que não ocorreu alavancagem
nas séries e que a variância condicional foi assimétrica nos retornos do
etanol, do café, do algodão, do boi gordo e do bezerro. Acharam também
que as volatilidades mais intensas, embora com convergência à sua média
histórica, ocorreram nos retornos do açúcar, da soja, do café, do trigo,
do frango e do boi gordo. As maiores volatilidades incondicionais
encontradas pelos autores foram dos retornos do etanol, do frango, do
algodão, da soja e do açúcar.

Conforme identificado na revisão de literatura, os trabalhos sobre a
mercado de derivativos de \emph{commodities} usam predominantemente
dados semanais e mensais, sendo que dados de alta frequência com
peridiocidade intradiária não ofi feito até então para a economia
brasileira. Conforme Moreira e Lemgruber (2004) a utilização de dados
intradiários melhor o ajuste do modelo e a previsibilidade da série
estudada, já que estes dados incorporam mais informações sobre a
microestrutura do mercado financeiro. E segundo Vicente et al. (2014) os
dados diários e consequentemente dados semanais e mensais, não capturam
todo o risco e incerteza que o agente econômico está expostos ao lidar
com instrumentos financeiros. Com isso, justifica-se um estudo que use
dados com informações intradiárias para estudar o mercado de
\emph{commodities}, sendo que assim contribuições sobre a existência ou
não de alavancagem financeira, pode ser feitas, já qe os estudos feitos
até então, não utilizaram esta peridiocidade dos dados.

\pagebreak

\begin{longtable}[]{@{}llll@{}}
\caption{Resumo dos estudos de volatilidade no mercado financeiro
brasileiro.}\tabularnewline
\toprule
\begin{minipage}[b]{0.17\columnwidth}\raggedright\strut
Autores\strut
\end{minipage} & \begin{minipage}[b]{0.17\columnwidth}\raggedright\strut
Objeto\strut
\end{minipage} & \begin{minipage}[b]{0.15\columnwidth}\raggedright\strut
Periodicidade\strut
\end{minipage} & \begin{minipage}[b]{0.23\columnwidth}\raggedright\strut
Método\strut
\end{minipage}\tabularnewline
\midrule
\endfirsthead
\toprule
\begin{minipage}[b]{0.17\columnwidth}\raggedright\strut
Autores\strut
\end{minipage} & \begin{minipage}[b]{0.17\columnwidth}\raggedright\strut
Objeto\strut
\end{minipage} & \begin{minipage}[b]{0.15\columnwidth}\raggedright\strut
Periodicidade\strut
\end{minipage} & \begin{minipage}[b]{0.23\columnwidth}\raggedright\strut
Método\strut
\end{minipage}\tabularnewline
\midrule
\endhead
\begin{minipage}[t]{0.17\columnwidth}\raggedright\strut
Moreira e Lemgruber (2004)\strut
\end{minipage} & \begin{minipage}[t]{0.17\columnwidth}\raggedright\strut
IBOVESPA\strut
\end{minipage} & \begin{minipage}[t]{0.15\columnwidth}\raggedright\strut
15 min e 1 dia\strut
\end{minipage} & \begin{minipage}[t]{0.23\columnwidth}\raggedright\strut
GARCH, EGARCH\strut
\end{minipage}\tabularnewline
\begin{minipage}[t]{0.17\columnwidth}\raggedright\strut
Carvalho et al. (2006)\strut
\end{minipage} & \begin{minipage}[t]{0.17\columnwidth}\raggedright\strut
Top 5 empresas IBOVESPA\strut
\end{minipage} & \begin{minipage}[t]{0.15\columnwidth}\raggedright\strut
15 min\strut
\end{minipage} & \begin{minipage}[t]{0.23\columnwidth}\raggedright\strut
EWMA, GARCH, EGARCH, GJR-GARCH, log retorno padronizado\strut
\end{minipage}\tabularnewline
\begin{minipage}[t]{0.17\columnwidth}\raggedright\strut
Ceretta et al. (2011)\strut
\end{minipage} & \begin{minipage}[t]{0.17\columnwidth}\raggedright\strut
IBOVESPA\strut
\end{minipage} & \begin{minipage}[t]{0.15\columnwidth}\raggedright\strut
15 min\strut
\end{minipage} & \begin{minipage}[t]{0.23\columnwidth}\raggedright\strut
APARCH\strut
\end{minipage}\tabularnewline
\begin{minipage}[t]{0.17\columnwidth}\raggedright\strut
Junior e Pereira (2013)\strut
\end{minipage} & \begin{minipage}[t]{0.17\columnwidth}\raggedright\strut
Top 5 empresas IBOVESPA\strut
\end{minipage} & \begin{minipage}[t]{0.15\columnwidth}\raggedright\strut
5, 15 e 30 min\strut
\end{minipage} & \begin{minipage}[t]{0.23\columnwidth}\raggedright\strut
HAR-RV e MIDAS-RV\strut
\end{minipage}\tabularnewline
\begin{minipage}[t]{0.17\columnwidth}\raggedright\strut
Morais et al. (2014)\strut
\end{minipage} & \begin{minipage}[t]{0.17\columnwidth}\raggedright\strut
Top 2 BMF\&BOVESPA\strut
\end{minipage} & \begin{minipage}[t]{0.15\columnwidth}\raggedright\strut
5 min\strut
\end{minipage} & \begin{minipage}[t]{0.23\columnwidth}\raggedright\strut
GARCH, EGARCH, CGARCH e TGARCH\strut
\end{minipage}\tabularnewline
\begin{minipage}[t]{0.17\columnwidth}\raggedright\strut
Santos e Ziegelmann (2014)\strut
\end{minipage} & \begin{minipage}[t]{0.17\columnwidth}\raggedright\strut
IBOVESPA\strut
\end{minipage} & \begin{minipage}[t]{0.15\columnwidth}\raggedright\strut
15 min\strut
\end{minipage} & \begin{minipage}[t]{0.23\columnwidth}\raggedright\strut
HAR, MIDAS e combinação HAR-MIDAS\strut
\end{minipage}\tabularnewline
\begin{minipage}[t]{0.17\columnwidth}\raggedright\strut
Vicente et al. (2014)\strut
\end{minipage} & \begin{minipage}[t]{0.17\columnwidth}\raggedright\strut
84 empresas da BMF\&BOVESPA\strut
\end{minipage} & \begin{minipage}[t]{0.15\columnwidth}\raggedright\strut
Amostragem aleatória de preços em um intervalo\strut
\end{minipage} & \begin{minipage}[t]{0.23\columnwidth}\raggedright\strut
Análise exploratória da Volatilidade Realizada\strut
\end{minipage}\tabularnewline
\begin{minipage}[t]{0.17\columnwidth}\raggedright\strut
Ziegelmann, Borges e Caldeira (2014)\strut
\end{minipage} & \begin{minipage}[t]{0.17\columnwidth}\raggedright\strut
Índice de 30 ações BMF\&BOVESPA\strut
\end{minipage} & \begin{minipage}[t]{0.15\columnwidth}\raggedright\strut
5 a 120 min\strut
\end{minipage} & \begin{minipage}[t]{0.23\columnwidth}\raggedright\strut
Covariância realizada, vt\_VECH escalar e MRK\strut
\end{minipage}\tabularnewline
\begin{minipage}[t]{0.17\columnwidth}\raggedright\strut
Araujo e Montini (2015)\strut
\end{minipage} & \begin{minipage}[t]{0.17\columnwidth}\raggedright\strut
IBOVESPA\strut
\end{minipage} & \begin{minipage}[t]{0.15\columnwidth}\raggedright\strut
5 min\strut
\end{minipage} & \begin{minipage}[t]{0.23\columnwidth}\raggedright\strut
HAR\_RV, MIDAS-RV e combinação HAR-MIDAS\strut
\end{minipage}\tabularnewline
\begin{minipage}[t]{0.17\columnwidth}\raggedright\strut
Silva et al. (2005)\strut
\end{minipage} & \begin{minipage}[t]{0.17\columnwidth}\raggedright\strut
café e soja\strut
\end{minipage} & \begin{minipage}[t]{0.15\columnwidth}\raggedright\strut
mensal\strut
\end{minipage} & \begin{minipage}[t]{0.23\columnwidth}\raggedright\strut
GARCH, EGARCH e TARCH\strut
\end{minipage}\tabularnewline
\begin{minipage}[t]{0.17\columnwidth}\raggedright\strut
Lima, Góis e Ulises (2007)\strut
\end{minipage} & \begin{minipage}[t]{0.17\columnwidth}\raggedright\strut
Açúcar, café, boi gordo, milho e soja\strut
\end{minipage} & \begin{minipage}[t]{0.15\columnwidth}\raggedright\strut
???\strut
\end{minipage} & \begin{minipage}[t]{0.23\columnwidth}\raggedright\strut
ARMA, ARIMA e ARFIMA; Arch e GARCH\strut
\end{minipage}\tabularnewline
\begin{minipage}[t]{0.17\columnwidth}\raggedright\strut
Silva (2008)\strut
\end{minipage} & \begin{minipage}[t]{0.17\columnwidth}\raggedright\strut
boi gordo\strut
\end{minipage} & \begin{minipage}[t]{0.15\columnwidth}\raggedright\strut
diária\strut
\end{minipage} & \begin{minipage}[t]{0.23\columnwidth}\raggedright\strut
ARCH, GARCH, EGARCH e TGARCH\strut
\end{minipage}\tabularnewline
\begin{minipage}[t]{0.17\columnwidth}\raggedright\strut
Teixeira et al. (2008)\strut
\end{minipage} & \begin{minipage}[t]{0.17\columnwidth}\raggedright\strut
cacau, café e boi gordo\strut
\end{minipage} & \begin{minipage}[t]{0.15\columnwidth}\raggedright\strut
diária\strut
\end{minipage} & \begin{minipage}[t]{0.23\columnwidth}\raggedright\strut
GARCH, EGARCH, TARCH\strut
\end{minipage}\tabularnewline
\begin{minipage}[t]{0.17\columnwidth}\raggedright\strut
Pereira et al. (2010)\strut
\end{minipage} & \begin{minipage}[t]{0.17\columnwidth}\raggedright\strut
Soja, café e boi gordo\strut
\end{minipage} & \begin{minipage}[t]{0.15\columnwidth}\raggedright\strut
semanal\strut
\end{minipage} & \begin{minipage}[t]{0.23\columnwidth}\raggedright\strut
EGARCH, TGARCH e GARCH-M\strut
\end{minipage}\tabularnewline
\begin{minipage}[t]{0.17\columnwidth}\raggedright\strut
Bodra (2012)\strut
\end{minipage} & \begin{minipage}[t]{0.17\columnwidth}\raggedright\strut
milho e soja\strut
\end{minipage} & \begin{minipage}[t]{0.15\columnwidth}\raggedright\strut
mensal\strut
\end{minipage} & \begin{minipage}[t]{0.23\columnwidth}\raggedright\strut
Volatilidade estocástica e saltos\strut
\end{minipage}\tabularnewline
\begin{minipage}[t]{0.17\columnwidth}\raggedright\strut
Freitas et al. (2015)\strut
\end{minipage} & \begin{minipage}[t]{0.17\columnwidth}\raggedright\strut
Açúcar, soja, milho, café, algodão, arroz, trigo, frango, boi gordo e
bezerro\strut
\end{minipage} & \begin{minipage}[t]{0.15\columnwidth}\raggedright\strut
semanal\strut
\end{minipage} & \begin{minipage}[t]{0.23\columnwidth}\raggedright\strut
APARCH\strut
\end{minipage}\tabularnewline
\bottomrule
\end{longtable}

Fonte: Elaboração dos autores

\pagebreak

\section*{Referências}\label{referencias}
\addcontentsline{toc}{section}{Referências}

\setlength{\parindent}{0in}

\hypertarget{refs}{}
\hypertarget{ref-araujo_estimacao_2015}{}
ARAUJO, A. C. DE; MONTINI, A. DE Á. Estimação da volatilidade percebida
futura por meio de combinação de projeções. \textbf{Anais}, 2015.
Disponível em:
\textless{}\url{http://bdpi.usp.br/single.php?_id=002723087}\textgreater{}.
Acesso em: 12/12/2016.

\hypertarget{ref-bodra_modelo_2012}{}
BODRA, R. A. Modelo de volatilidade estocástica com saltos aplicado a
commodities agrícolas., 2012. Disponível em:
\textless{}\url{http://bibliotecadigital.fgv.br/dspace/handle/10438/10374}\textgreater{}.
Acesso em: 29/12/2016.

\hypertarget{ref-carvalho_modeling_2006}{}
CARVALHO, M. C.; FREIRE, M. A. S.; MEDEIROS, M. C.; SOUZA, L. R.
Modeling and Forecasting the Volatility of Brazilian Asset Returns: a
Realized Variance Approach. \textbf{Brazilian Review of Finance}, v. 4,
n. 1, p. 55--77, 2006. Disponível em:
\textless{}\url{http://bibliotecadigital.fgv.br/ojs/index.php/rbfin/article/view/1155}\textgreater{}.
Acesso em: 12/12/2016.

\hypertarget{ref-ceretta_intraday_2011}{}
CERETTA, P. S.; BARBA, F. G. DE; VIEIRA, K. M.; CASARIN, F. Intraday
volatility forecasting: analysis of alternative distributions.
\textbf{Brazilian Review of Finance}, v. 9, n. 2, p. 209--226, 2011.
Disponível em:
\textless{}\url{http://bibliotecadigital.fgv.br/ojs/index.php/rbfin/article/view/2586}\textgreater{}.
Acesso em: 12/12/2016.

\hypertarget{ref-freitas_volatilidade_2015}{}
FREITAS, C. A. DE; SÁFADI, T.; FREITAS, C. A. DE; SÁFADI, T.
Volatilidade dos Retornos de Commodities Agropecuárias Brasileiras: um
teste utilizando o modelo APARCH. \textbf{Revista de Economia e
Sociologia Rural}, v. 53, n. 2, p. 211--228, 2015. Disponível em:
\textless{}\url{http://www.scielo.br/scielo.php?script=sci_abstract\&pid=S0103-20032015000200211\&lng=en\&nrm=iso\&tlng=pt}\textgreater{}.
Acesso em: 29/12/2016.

\hypertarget{ref-junior_modeling_2013}{}
JUNIOR, M. V. W.; PEREIRA, P. L. V. Modeling and Forecasting of Realized
Volatility: Evidence from Brazil. \textbf{Brazilian Review of
Econometrics}, v. 31, n. 2, p. 315--337, 2013. Disponível em:
\textless{}\url{http://bibliotecadigital.fgv.br/ojs/index.php/bre/article/view/4056}\textgreater{}.
Acesso em: 12/12/2016.

\hypertarget{ref-lima_previsao_2007}{}
LIMA, R. C.; GÓIS, M. R.; ULISES, C. Previsão de preços futuros de
Commodities agrícolas com diferenciações inteira e fracionária, e erros
heteroscedásticos. \textbf{Revista de Economia e Sociologia Rural}, v.
45, n. 3, p. 621--644, 2007. Disponível em:
\textless{}\url{http://www.scielo.br/scielo.php?script=sci_abstract\&pid=S0103-20032007000300004\&lng=en\&nrm=iso\&tlng=es}\textgreater{}.
Acesso em: 29/12/2016.

\hypertarget{ref-morais_estimating_2014}{}
MORAIS, O.; VAL, F. DE F.; PINTO, A. C. F.; KL, M. C.; MACELLY.
Estimating the Volatility of Brazilian Equities using Garch-Type Models
and High-Frequency Volatility Measures. \textbf{Global Journal of
Management And Business Research}, v. 14, n. 5, 2014. Disponível em:
\textless{}\url{http://journalofbusiness.org/index.php/GJMBR/article/view/1520}\textgreater{}.
Acesso em: 12/12/2016.

\hypertarget{ref-moreira_o_2004}{}
MOREIRA, J. M. DE S.; LEMGRUBER, E. F. O uso de dados de alta freqüência
na estimação da volatilidade e do valor em risco para o IBOVESPA.
\textbf{Revista Brasileira de Economia}, v. 58, n. 1, p. 100--120, 2004.
Disponível em:
\textless{}\url{http://www.scielo.br/scielo.php?script=sci_abstract\&pid=S0034-71402004000100005\&lng=en\&nrm=iso\&tlng=pt}\textgreater{}.
Acesso em: 12/12/2016.

\hypertarget{ref-pereira_volatilidade_2010}{}
PEREIRA, V. DA F.; LIMA, J. E. DE; BRAGA, M. J.; MENDONÇA, T. G. DE.
Volatilidade condicional dos retornos de commodities agropecuárias
brasileiras seguidos pela soja e pelo boi gordo. \textbf{Revista de
Economia}, v. 36, n. 3, 2010. Disponível em:
\textless{}\url{http://revistas.ufpr.br/economia/article/view/14058}\textgreater{}.
Acesso em: 29/12/2016.

\hypertarget{ref-santos_volatility_2014}{}
SANTOS, D. G.; ZIEGELMANN, F. A. Volatility Forecasting via MIDAS, HAR
and their Combination: An Empirical Comparative Study for IBOVESPA.
\textbf{Journal of Forecasting}, v. 33, n. 4, p. 284--299, 2014.
Disponível em:
\textless{}\url{http://onlinelibrary.wiley.com/doi/10.1002/for.2287/abstract}\textgreater{}.
Acesso em: 12/12/2016.

\hypertarget{ref-silva_alise_2008}{}
SILVA, C. A. G. DA. \textbf{ANÁLISE DA VOLATILIDADE DOS PREÇOS DE BOI
GORDO NO ESTADO DE SÃO PAULO: UMA APLICAÇÃO DOS MODELOS GARCH}. 46th
Congress, July 20-23, 2008, Rio Branco, Acre, Brasil, Sociedade
Brasileira de Economia, Administracao e Sociologia Rural (SOBER), 2008.

\hypertarget{ref-silva_uma_2005}{}
SILVA, W. S. DA; SÁFADI, T.; JÚNIOR, C.; DE, L. G. Uma análise empírica
da volatilidade do retorno de commodities agrícolas utilizando modelos
ARCH: os casos do café e da soja. \textbf{Revista de Economia e
Sociologia Rural}, v. 43, n. 1, p. 119--134, 2005. Disponível em:
\textless{}\url{http://www.scielo.br/scielo.php?script=sci_abstract\&pid=S0103-20032005000100007\&lng=en\&nrm=iso\&tlng=pt}\textgreater{}.
Acesso em: 29/12/2016.

\hypertarget{ref-teixeira_dinamica_2008}{}
TEIXEIRA, G. DA S.; MAIA, S. F.; FIGUEIREDO, N. M.; PEREIRA, E. S.;
PINTO, P. A. L. D. A. \textbf{Dinâmica Da Volatilidade Do Retorno Das
Principais Commodities Brasileiras: Uma Abordagem Dos Modelos Arch}.
46th Congress, July 20-23, 2008, Rio Branco, Acre, Brasil, Sociedade
Brasileira de Economia, Administracao e Sociologia Rural (SOBER), 2008.

\hypertarget{ref-vicente_assessing_2014}{}
VICENTE, J. V. M.; ARAUJO, G. S.; CASTRO, P. B. F. DE; TAVARES, F. N.
Assessing Day-to-Day Volatility: Does the Trading Time Matter?
\textbf{Brazilian Review of Finance}, v. 12, n. 1, p. 41--66, 2014.
Disponível em:
\textless{}\url{http://bibliotecadigital.fgv.br/ojs/index.php/rbfin/article/view/13483}\textgreater{}.
Acesso em: 12/12/2016.

\hypertarget{ref-ziegelmann_selection_2014}{}
ZIEGELMANN, F. A.; BORGES, B.; CALDEIRA, J. F. Selection of Minimum
Variance Portfolio Using Intraday Data: An Empirical Comparison Among
Different Realized Measures for BM\&FBovespa Data. \textbf{Brazilian
Review of Econometrics}, v. 35, n. 1, p. 23--46, 2014. Disponível em:
\textless{}\url{http://bibliotecadigital.fgv.br/ojs/index.php/bre/article/view/21453}\textgreater{}.
Acesso em: 12/12/2016.


\end{document}

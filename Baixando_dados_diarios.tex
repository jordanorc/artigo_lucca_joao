\documentclass[]{article}
\usepackage{lmodern}
\usepackage{amssymb,amsmath}
\usepackage{ifxetex,ifluatex}
\usepackage{fixltx2e} % provides \textsubscript
\ifnum 0\ifxetex 1\fi\ifluatex 1\fi=0 % if pdftex
  \usepackage[T1]{fontenc}
  \usepackage[utf8]{inputenc}
\else % if luatex or xelatex
  \ifxetex
    \usepackage{mathspec}
  \else
    \usepackage{fontspec}
  \fi
  \defaultfontfeatures{Ligatures=TeX,Scale=MatchLowercase}
  \newcommand{\euro}{€}
\fi
% use upquote if available, for straight quotes in verbatim environments
\IfFileExists{upquote.sty}{\usepackage{upquote}}{}
% use microtype if available
\IfFileExists{microtype.sty}{%
\usepackage{microtype}
\UseMicrotypeSet[protrusion]{basicmath} % disable protrusion for tt fonts
}{}
\usepackage[margin=1in]{geometry}
\usepackage{hyperref}
\PassOptionsToPackage{usenames,dvipsnames}{color} % color is loaded by hyperref
\hypersetup{unicode=true,
            pdftitle={Baixando dados diários},
            pdfauthor={Lucca Simeoni Pavan  João Carlos de Carvalho},
            pdfborder={0 0 0},
            breaklinks=true}
\urlstyle{same}  % don't use monospace font for urls
\usepackage{color}
\usepackage{fancyvrb}
\newcommand{\VerbBar}{|}
\newcommand{\VERB}{\Verb[commandchars=\\\{\}]}
\DefineVerbatimEnvironment{Highlighting}{Verbatim}{commandchars=\\\{\}}
% Add ',fontsize=\small' for more characters per line
\usepackage{framed}
\definecolor{shadecolor}{RGB}{248,248,248}
\newenvironment{Shaded}{\begin{snugshade}}{\end{snugshade}}
\newcommand{\KeywordTok}[1]{\textcolor[rgb]{0.13,0.29,0.53}{\textbf{{#1}}}}
\newcommand{\DataTypeTok}[1]{\textcolor[rgb]{0.13,0.29,0.53}{{#1}}}
\newcommand{\DecValTok}[1]{\textcolor[rgb]{0.00,0.00,0.81}{{#1}}}
\newcommand{\BaseNTok}[1]{\textcolor[rgb]{0.00,0.00,0.81}{{#1}}}
\newcommand{\FloatTok}[1]{\textcolor[rgb]{0.00,0.00,0.81}{{#1}}}
\newcommand{\ConstantTok}[1]{\textcolor[rgb]{0.00,0.00,0.00}{{#1}}}
\newcommand{\CharTok}[1]{\textcolor[rgb]{0.31,0.60,0.02}{{#1}}}
\newcommand{\SpecialCharTok}[1]{\textcolor[rgb]{0.00,0.00,0.00}{{#1}}}
\newcommand{\StringTok}[1]{\textcolor[rgb]{0.31,0.60,0.02}{{#1}}}
\newcommand{\VerbatimStringTok}[1]{\textcolor[rgb]{0.31,0.60,0.02}{{#1}}}
\newcommand{\SpecialStringTok}[1]{\textcolor[rgb]{0.31,0.60,0.02}{{#1}}}
\newcommand{\ImportTok}[1]{{#1}}
\newcommand{\CommentTok}[1]{\textcolor[rgb]{0.56,0.35,0.01}{\textit{{#1}}}}
\newcommand{\DocumentationTok}[1]{\textcolor[rgb]{0.56,0.35,0.01}{\textbf{\textit{{#1}}}}}
\newcommand{\AnnotationTok}[1]{\textcolor[rgb]{0.56,0.35,0.01}{\textbf{\textit{{#1}}}}}
\newcommand{\CommentVarTok}[1]{\textcolor[rgb]{0.56,0.35,0.01}{\textbf{\textit{{#1}}}}}
\newcommand{\OtherTok}[1]{\textcolor[rgb]{0.56,0.35,0.01}{{#1}}}
\newcommand{\FunctionTok}[1]{\textcolor[rgb]{0.00,0.00,0.00}{{#1}}}
\newcommand{\VariableTok}[1]{\textcolor[rgb]{0.00,0.00,0.00}{{#1}}}
\newcommand{\ControlFlowTok}[1]{\textcolor[rgb]{0.13,0.29,0.53}{\textbf{{#1}}}}
\newcommand{\OperatorTok}[1]{\textcolor[rgb]{0.81,0.36,0.00}{\textbf{{#1}}}}
\newcommand{\BuiltInTok}[1]{{#1}}
\newcommand{\ExtensionTok}[1]{{#1}}
\newcommand{\PreprocessorTok}[1]{\textcolor[rgb]{0.56,0.35,0.01}{\textit{{#1}}}}
\newcommand{\AttributeTok}[1]{\textcolor[rgb]{0.77,0.63,0.00}{{#1}}}
\newcommand{\RegionMarkerTok}[1]{{#1}}
\newcommand{\InformationTok}[1]{\textcolor[rgb]{0.56,0.35,0.01}{\textbf{\textit{{#1}}}}}
\newcommand{\WarningTok}[1]{\textcolor[rgb]{0.56,0.35,0.01}{\textbf{\textit{{#1}}}}}
\newcommand{\AlertTok}[1]{\textcolor[rgb]{0.94,0.16,0.16}{{#1}}}
\newcommand{\ErrorTok}[1]{\textcolor[rgb]{0.64,0.00,0.00}{\textbf{{#1}}}}
\newcommand{\NormalTok}[1]{{#1}}
\usepackage{graphicx,grffile}
\makeatletter
\def\maxwidth{\ifdim\Gin@nat@width>\linewidth\linewidth\else\Gin@nat@width\fi}
\def\maxheight{\ifdim\Gin@nat@height>\textheight\textheight\else\Gin@nat@height\fi}
\makeatother
% Scale images if necessary, so that they will not overflow the page
% margins by default, and it is still possible to overwrite the defaults
% using explicit options in \includegraphics[width, height, ...]{}
\setkeys{Gin}{width=\maxwidth,height=\maxheight,keepaspectratio}
\setlength{\parindent}{0pt}
\setlength{\parskip}{6pt plus 2pt minus 1pt}
\setlength{\emergencystretch}{3em}  % prevent overfull lines
\providecommand{\tightlist}{%
  \setlength{\itemsep}{0pt}\setlength{\parskip}{0pt}}
\setcounter{secnumdepth}{5}

%%% Use protect on footnotes to avoid problems with footnotes in titles
\let\rmarkdownfootnote\footnote%
\def\footnote{\protect\rmarkdownfootnote}

%%% Change title format to be more compact
\usepackage{titling}

% Create subtitle command for use in maketitle
\newcommand{\subtitle}[1]{
  \posttitle{
    \begin{center}\large#1\end{center}
    }
}

\setlength{\droptitle}{-2em}
  \title{Baixando dados diários}
  \pretitle{\vspace{\droptitle}\centering\huge}
  \posttitle{\par}
  \author{Lucca Simeoni Pavan \hspace{1cm} João Carlos de Carvalho}
  \preauthor{\centering\large\emph}
  \postauthor{\par}
  \predate{\centering\large\emph}
  \postdate{\par}
  \date{\today}


\setlength\parindent{24pt}
\usepackage[english, brazil]{babel}

% Redefines (sub)paragraphs to behave more like sections
\ifx\paragraph\undefined\else
\let\oldparagraph\paragraph
\renewcommand{\paragraph}[1]{\oldparagraph{#1}\mbox{}}
\fi
\ifx\subparagraph\undefined\else
\let\oldsubparagraph\subparagraph
\renewcommand{\subparagraph}[1]{\oldsubparagraph{#1}\mbox{}}
\fi

\begin{document}
\maketitle

{
\setcounter{tocdepth}{2}
\tableofcontents
}
\begin{Shaded}
\begin{Highlighting}[]
\NormalTok{knitr::opts_chunk$}\KeywordTok{set}\NormalTok{(}\DataTypeTok{echo =} \OtherTok{TRUE}\NormalTok{, }\DataTypeTok{cache =} \OtherTok{TRUE}\NormalTok{, }\DataTypeTok{warning =} \OtherTok{FALSE}\NormalTok{, }\DataTypeTok{message =} \OtherTok{FALSE}\NormalTok{, }
    \DataTypeTok{error =} \OtherTok{FALSE}\NormalTok{, }\DataTypeTok{tidy =} \OtherTok{TRUE}\NormalTok{, }\DataTypeTok{tidy.opts =} \KeywordTok{list}\NormalTok{(}\DataTypeTok{width.cutoff =} \DecValTok{70}\NormalTok{))}
\end{Highlighting}
\end{Shaded}

\section{Ranking de negociações}\label{ranking-de-negociacoes}

\begin{Shaded}
\begin{Highlighting}[]
\KeywordTok{library}\NormalTok{(GetHFData)}
\NormalTok{tickers_equity <-}\StringTok{ }\KeywordTok{ghfd_get_available_tickers_from_ftp}\NormalTok{(}\DataTypeTok{my.date =} \StringTok{"2016-10-30"}\NormalTok{, }
    \DataTypeTok{type.market =} \StringTok{"equity"}\NormalTok{, }\DataTypeTok{max.dl.tries =} \DecValTok{10}\NormalTok{)}
\end{Highlighting}
\end{Shaded}

\begin{verbatim}
## 
## Reading ftp contents for equity (attempt = 1|10) Attempt 1 - File exists, skipping dl
\end{verbatim}

\begin{Shaded}
\begin{Highlighting}[]
\KeywordTok{head}\NormalTok{(tickers_equity, }\DataTypeTok{n =} \DecValTok{10}\NormalTok{)}
\end{Highlighting}
\end{Shaded}

\begin{verbatim}
##    tickers n.trades                     f.name
## 1    ITSA4    56578 ftp files/NEG_20160930.zip
## 2    PETR4    29123 ftp files/NEG_20160930.zip
## 3    ITUB4    24040 ftp files/NEG_20160930.zip
## 4    BBDC4    21710 ftp files/NEG_20160930.zip
## 5    ABEV3    20719 ftp files/NEG_20160930.zip
## 6    BBSE3    20450 ftp files/NEG_20160930.zip
## 7    BVMF3    19170 ftp files/NEG_20160930.zip
## 8    GOAU4    16868 ftp files/NEG_20160930.zip
## 9    LREN3    16716 ftp files/NEG_20160930.zip
## 10   VALE5    16141 ftp files/NEG_20160930.zip
\end{verbatim}

Criando um vetor com as 6 ações mais negociadas em 30/10/2016.

\begin{Shaded}
\begin{Highlighting}[]
\NormalTok{top_6 <-}\StringTok{ }\KeywordTok{c}\NormalTok{(}\KeywordTok{as.character}\NormalTok{(}\KeywordTok{head}\NormalTok{(tickers_equity$tickers)))}
\KeywordTok{print}\NormalTok{(top_6)}
\end{Highlighting}
\end{Shaded}

\begin{verbatim}
## [1] "ITSA4" "PETR4" "ITUB4" "BBDC4" "ABEV3" "BBSE3"
\end{verbatim}

Baixando os dados

\begin{Shaded}
\begin{Highlighting}[]
\NormalTok{dados_top6 <-}\StringTok{ }\KeywordTok{ghfd_get_HF_data}\NormalTok{(top_6, }\DataTypeTok{type.market =} \StringTok{"equity"}\NormalTok{, }\DataTypeTok{first.date =} \KeywordTok{as.Date}\NormalTok{(}\StringTok{"2014-11-03"}\NormalTok{), }
    \DataTypeTok{last.date =} \KeywordTok{as.Date}\NormalTok{(}\StringTok{"2016-10-30"}\NormalTok{), }\DataTypeTok{first.time =} \StringTok{"9:00:00"}\NormalTok{, }\DataTypeTok{last.time =} \StringTok{"18:00:00"}\NormalTok{, }
    \DataTypeTok{type.output =} \StringTok{"agg"}\NormalTok{, }\DataTypeTok{agg.diff =} \StringTok{"1 hour"}\NormalTok{, }\DataTypeTok{dl.dir =} \StringTok{"ftp files"}\NormalTok{, }\DataTypeTok{max.dl.tries =} \DecValTok{10}\NormalTok{, }
    \DataTypeTok{clean.files =} \OtherTok{FALSE}\NormalTok{)}
\KeywordTok{save}\NormalTok{(dados_top6, }\DataTypeTok{file =} \StringTok{"dados_top6.Rda"}\NormalTok{)}
\KeywordTok{head}\NormalTok{(dados_top6, }\DataTypeTok{n =} \DecValTok{6}\NormalTok{)}
\end{Highlighting}
\end{Shaded}

\begin{Shaded}
\begin{Highlighting}[]
\KeywordTok{load}\NormalTok{(}\StringTok{"dados_top6.Rda"}\NormalTok{)}
\KeywordTok{dim}\NormalTok{(dados_top6)}
\end{Highlighting}
\end{Shaded}

\begin{verbatim}
## [1] 22667    13
\end{verbatim}

Agora irei criar um banco de dados para cada ação e depois tornar os
preços diários.

\begin{Shaded}
\begin{Highlighting}[]
\KeywordTok{library}\NormalTok{(dplyr)}
\NormalTok{dados_ITSA4 <-}\StringTok{ }\KeywordTok{filter}\NormalTok{(dados_top6, InstrumentSymbol ==}\StringTok{ "ITSA4"}\NormalTok{) %>%}\StringTok{ }\KeywordTok{select}\NormalTok{(SessionDate, }
    \NormalTok{weighted.price)}
\NormalTok{dados_PETR4 <-}\StringTok{ }\KeywordTok{filter}\NormalTok{(dados_top6, InstrumentSymbol ==}\StringTok{ "PETR4"}\NormalTok{) %>%}\StringTok{ }\KeywordTok{select}\NormalTok{(SessionDate, }
    \NormalTok{weighted.price)}
\NormalTok{dados_ITUB4 <-}\StringTok{ }\KeywordTok{filter}\NormalTok{(dados_top6, InstrumentSymbol ==}\StringTok{ "ITUB4"}\NormalTok{) %>%}\StringTok{ }\KeywordTok{select}\NormalTok{(SessionDate, }
    \NormalTok{weighted.price)}
\NormalTok{dados_BBDC4 <-}\StringTok{ }\KeywordTok{filter}\NormalTok{(dados_top6, InstrumentSymbol ==}\StringTok{ "BBDC4"}\NormalTok{) %>%}\StringTok{ }\KeywordTok{select}\NormalTok{(SessionDate, }
    \NormalTok{weighted.price)}
\NormalTok{dados_ABEV3 <-}\StringTok{ }\KeywordTok{filter}\NormalTok{(dados_top6, InstrumentSymbol ==}\StringTok{ "ABEV3"}\NormalTok{) %>%}\StringTok{ }\KeywordTok{select}\NormalTok{(SessionDate, }
    \NormalTok{weighted.price)}
\NormalTok{dados_BBSE3 <-}\StringTok{ }\KeywordTok{filter}\NormalTok{(dados_top6, InstrumentSymbol ==}\StringTok{ "BBSE3"}\NormalTok{) %>%}\StringTok{ }\KeywordTok{select}\NormalTok{(SessionDate, }
    \NormalTok{weighted.price)}
\end{Highlighting}
\end{Shaded}

\hypertarget{refs}{}

\end{document}

\documentclass[]{article}
\usepackage{lmodern}
\usepackage{amssymb,amsmath}
\usepackage{ifxetex,ifluatex}
\usepackage{fixltx2e} % provides \textsubscript
\ifnum 0\ifxetex 1\fi\ifluatex 1\fi=0 % if pdftex
  \usepackage[T1]{fontenc}
  \usepackage[utf8]{inputenc}
\else % if luatex or xelatex
  \ifxetex
    \usepackage{mathspec}
  \else
    \usepackage{fontspec}
  \fi
  \defaultfontfeatures{Ligatures=TeX,Scale=MatchLowercase}
  \newcommand{\euro}{€}
\fi
% use upquote if available, for straight quotes in verbatim environments
\IfFileExists{upquote.sty}{\usepackage{upquote}}{}
% use microtype if available
\IfFileExists{microtype.sty}{%
\usepackage{microtype}
\UseMicrotypeSet[protrusion]{basicmath} % disable protrusion for tt fonts
}{}
\usepackage[margin=1in]{geometry}
\usepackage{hyperref}
\PassOptionsToPackage{usenames,dvipsnames}{color} % color is loaded by hyperref
\hypersetup{unicode=true,
            pdftitle={Baixando dados diários},
            pdfauthor={Lucca Simeoni Pavan  João Carlos de Carvalho},
            pdfborder={0 0 0},
            breaklinks=true}
\urlstyle{same}  % don't use monospace font for urls
\usepackage{color}
\usepackage{fancyvrb}
\newcommand{\VerbBar}{|}
\newcommand{\VERB}{\Verb[commandchars=\\\{\}]}
\DefineVerbatimEnvironment{Highlighting}{Verbatim}{commandchars=\\\{\}}
% Add ',fontsize=\small' for more characters per line
\usepackage{framed}
\definecolor{shadecolor}{RGB}{248,248,248}
\newenvironment{Shaded}{\begin{snugshade}}{\end{snugshade}}
\newcommand{\KeywordTok}[1]{\textcolor[rgb]{0.13,0.29,0.53}{\textbf{{#1}}}}
\newcommand{\DataTypeTok}[1]{\textcolor[rgb]{0.13,0.29,0.53}{{#1}}}
\newcommand{\DecValTok}[1]{\textcolor[rgb]{0.00,0.00,0.81}{{#1}}}
\newcommand{\BaseNTok}[1]{\textcolor[rgb]{0.00,0.00,0.81}{{#1}}}
\newcommand{\FloatTok}[1]{\textcolor[rgb]{0.00,0.00,0.81}{{#1}}}
\newcommand{\ConstantTok}[1]{\textcolor[rgb]{0.00,0.00,0.00}{{#1}}}
\newcommand{\CharTok}[1]{\textcolor[rgb]{0.31,0.60,0.02}{{#1}}}
\newcommand{\SpecialCharTok}[1]{\textcolor[rgb]{0.00,0.00,0.00}{{#1}}}
\newcommand{\StringTok}[1]{\textcolor[rgb]{0.31,0.60,0.02}{{#1}}}
\newcommand{\VerbatimStringTok}[1]{\textcolor[rgb]{0.31,0.60,0.02}{{#1}}}
\newcommand{\SpecialStringTok}[1]{\textcolor[rgb]{0.31,0.60,0.02}{{#1}}}
\newcommand{\ImportTok}[1]{{#1}}
\newcommand{\CommentTok}[1]{\textcolor[rgb]{0.56,0.35,0.01}{\textit{{#1}}}}
\newcommand{\DocumentationTok}[1]{\textcolor[rgb]{0.56,0.35,0.01}{\textbf{\textit{{#1}}}}}
\newcommand{\AnnotationTok}[1]{\textcolor[rgb]{0.56,0.35,0.01}{\textbf{\textit{{#1}}}}}
\newcommand{\CommentVarTok}[1]{\textcolor[rgb]{0.56,0.35,0.01}{\textbf{\textit{{#1}}}}}
\newcommand{\OtherTok}[1]{\textcolor[rgb]{0.56,0.35,0.01}{{#1}}}
\newcommand{\FunctionTok}[1]{\textcolor[rgb]{0.00,0.00,0.00}{{#1}}}
\newcommand{\VariableTok}[1]{\textcolor[rgb]{0.00,0.00,0.00}{{#1}}}
\newcommand{\ControlFlowTok}[1]{\textcolor[rgb]{0.13,0.29,0.53}{\textbf{{#1}}}}
\newcommand{\OperatorTok}[1]{\textcolor[rgb]{0.81,0.36,0.00}{\textbf{{#1}}}}
\newcommand{\BuiltInTok}[1]{{#1}}
\newcommand{\ExtensionTok}[1]{{#1}}
\newcommand{\PreprocessorTok}[1]{\textcolor[rgb]{0.56,0.35,0.01}{\textit{{#1}}}}
\newcommand{\AttributeTok}[1]{\textcolor[rgb]{0.77,0.63,0.00}{{#1}}}
\newcommand{\RegionMarkerTok}[1]{{#1}}
\newcommand{\InformationTok}[1]{\textcolor[rgb]{0.56,0.35,0.01}{\textbf{\textit{{#1}}}}}
\newcommand{\WarningTok}[1]{\textcolor[rgb]{0.56,0.35,0.01}{\textbf{\textit{{#1}}}}}
\newcommand{\AlertTok}[1]{\textcolor[rgb]{0.94,0.16,0.16}{{#1}}}
\newcommand{\ErrorTok}[1]{\textcolor[rgb]{0.64,0.00,0.00}{\textbf{{#1}}}}
\newcommand{\NormalTok}[1]{{#1}}
\usepackage{graphicx,grffile}
\makeatletter
\def\maxwidth{\ifdim\Gin@nat@width>\linewidth\linewidth\else\Gin@nat@width\fi}
\def\maxheight{\ifdim\Gin@nat@height>\textheight\textheight\else\Gin@nat@height\fi}
\makeatother
% Scale images if necessary, so that they will not overflow the page
% margins by default, and it is still possible to overwrite the defaults
% using explicit options in \includegraphics[width, height, ...]{}
\setkeys{Gin}{width=\maxwidth,height=\maxheight,keepaspectratio}
\setlength{\parindent}{0pt}
\setlength{\parskip}{6pt plus 2pt minus 1pt}
\setlength{\emergencystretch}{3em}  % prevent overfull lines
\providecommand{\tightlist}{%
  \setlength{\itemsep}{0pt}\setlength{\parskip}{0pt}}
\setcounter{secnumdepth}{5}

%%% Use protect on footnotes to avoid problems with footnotes in titles
\let\rmarkdownfootnote\footnote%
\def\footnote{\protect\rmarkdownfootnote}

%%% Change title format to be more compact
\usepackage{titling}

% Create subtitle command for use in maketitle
\newcommand{\subtitle}[1]{
  \posttitle{
    \begin{center}\large#1\end{center}
    }
}

\setlength{\droptitle}{-2em}
  \title{Baixando dados diários}
  \pretitle{\vspace{\droptitle}\centering\huge}
  \posttitle{\par}
  \author{Lucca Simeoni Pavan \hspace{1cm} João Carlos de Carvalho}
  \preauthor{\centering\large\emph}
  \postauthor{\par}
  \predate{\centering\large\emph}
  \postdate{\par}
  \date{\today}


\setlength\parindent{24pt}
\usepackage[english, brazil]{babel}

% Redefines (sub)paragraphs to behave more like sections
\ifx\paragraph\undefined\else
\let\oldparagraph\paragraph
\renewcommand{\paragraph}[1]{\oldparagraph{#1}\mbox{}}
\fi
\ifx\subparagraph\undefined\else
\let\oldsubparagraph\subparagraph
\renewcommand{\subparagraph}[1]{\oldsubparagraph{#1}\mbox{}}
\fi

\begin{document}
\maketitle

{
\setcounter{tocdepth}{2}
\tableofcontents
}
\begin{Shaded}
\begin{Highlighting}[]
\NormalTok{knitr::opts_chunk$}\KeywordTok{set}\NormalTok{(}\DataTypeTok{echo =} \OtherTok{TRUE}\NormalTok{, }\DataTypeTok{cache =} \OtherTok{FALSE}\NormalTok{, }\DataTypeTok{warning =} \OtherTok{FALSE}\NormalTok{, }\DataTypeTok{message =} \OtherTok{FALSE}\NormalTok{, }
    \DataTypeTok{error =} \OtherTok{FALSE}\NormalTok{, }\DataTypeTok{tidy =} \OtherTok{TRUE}\NormalTok{, }\DataTypeTok{tidy.opts =} \KeywordTok{list}\NormalTok{(}\DataTypeTok{width.cutoff =} \DecValTok{70}\NormalTok{))}
\end{Highlighting}
\end{Shaded}

\section{Ranking de negociações}\label{ranking-de-negociacoes}

\begin{Shaded}
\begin{Highlighting}[]
\KeywordTok{library}\NormalTok{(GetHFData)}
\NormalTok{tickers_equity <-}\StringTok{ }\KeywordTok{ghfd_get_available_tickers_from_ftp}\NormalTok{(}\DataTypeTok{my.date =} \StringTok{"2016-10-30"}\NormalTok{, }
    \DataTypeTok{type.market =} \StringTok{"equity"}\NormalTok{, }\DataTypeTok{max.dl.tries =} \DecValTok{10}\NormalTok{)}
\end{Highlighting}
\end{Shaded}

\begin{verbatim}
## 
## Reading ftp contents for equity (attempt = 1|10) Attempt 1 - File exists, skipping dl
\end{verbatim}

\begin{Shaded}
\begin{Highlighting}[]
\KeywordTok{head}\NormalTok{(tickers_equity, }\DataTypeTok{n =} \DecValTok{10}\NormalTok{)}
\end{Highlighting}
\end{Shaded}

\begin{verbatim}
##    tickers n.trades                     f.name
## 1    PETR4    52393 ftp files/NEG_20161117.zip
## 2    JBSS3    45174 ftp files/NEG_20161117.zip
## 3    ITSA4    39200 ftp files/NEG_20161117.zip
## 4    ITUB4    30529 ftp files/NEG_20161117.zip
## 5    VALE5    30423 ftp files/NEG_20161117.zip
## 6    BVMF3    29099 ftp files/NEG_20161117.zip
## 7    BBDC4    26923 ftp files/NEG_20161117.zip
## 8    ABEV3    26786 ftp files/NEG_20161117.zip
## 9    BBAS3    26672 ftp files/NEG_20161117.zip
## 10   RUMO3    26274 ftp files/NEG_20161117.zip
\end{verbatim}

Criando um vetor com as 6 ações mais negociadas em 30/10/2016.

\begin{Shaded}
\begin{Highlighting}[]
\NormalTok{top_6 <-}\StringTok{ }\KeywordTok{c}\NormalTok{(}\KeywordTok{as.character}\NormalTok{(}\KeywordTok{head}\NormalTok{(tickers_equity$tickers)))}
\KeywordTok{print}\NormalTok{(top_6)}
\end{Highlighting}
\end{Shaded}

\begin{verbatim}
## [1] "PETR4" "JBSS3" "ITSA4" "ITUB4" "VALE5" "BVMF3"
\end{verbatim}

Baixando os dados

\begin{Shaded}
\begin{Highlighting}[]
\NormalTok{dados_top6 <-}\StringTok{ }\KeywordTok{ghfd_get_HF_data}\NormalTok{(top_6, }\DataTypeTok{type.market =} \StringTok{"equity"}\NormalTok{, }\DataTypeTok{first.date =} \KeywordTok{as.Date}\NormalTok{(}\StringTok{"2014-11-03"}\NormalTok{), }
    \DataTypeTok{last.date =} \KeywordTok{as.Date}\NormalTok{(}\StringTok{"2016-10-30"}\NormalTok{), }\DataTypeTok{first.time =} \StringTok{"9:00:00"}\NormalTok{, }\DataTypeTok{last.time =} \StringTok{"18:00:00"}\NormalTok{, }
    \DataTypeTok{type.output =} \StringTok{"agg"}\NormalTok{, }\DataTypeTok{agg.diff =} \StringTok{"1 hour"}\NormalTok{, }\DataTypeTok{dl.dir =} \StringTok{"ftp files"}\NormalTok{, }\DataTypeTok{max.dl.tries =} \DecValTok{10}\NormalTok{, }
    \DataTypeTok{clean.files =} \OtherTok{FALSE}\NormalTok{)}
\KeywordTok{save}\NormalTok{(dados_top6, }\DataTypeTok{file =} \StringTok{"dados_top6.Rda"}\NormalTok{)}
\KeywordTok{head}\NormalTok{(dados_top6, }\DataTypeTok{n =} \DecValTok{6}\NormalTok{)}
\end{Highlighting}
\end{Shaded}

\begin{Shaded}
\begin{Highlighting}[]
\KeywordTok{load}\NormalTok{(}\StringTok{"dados_top6.Rda"}\NormalTok{)}
\KeywordTok{dim}\NormalTok{(dados_top6)}
\end{Highlighting}
\end{Shaded}

\begin{verbatim}
## [1] 22667    13
\end{verbatim}

\begin{Shaded}
\begin{Highlighting}[]
\KeywordTok{str}\NormalTok{(dados_top6)}
\end{Highlighting}
\end{Shaded}

\begin{verbatim}
## 'data.frame':    22667 obs. of  13 variables:
##  $ InstrumentSymbol: chr  "ABEV3" "ABEV3" "ABEV3" "ABEV3" ...
##  $ SessionDate     : Date, format: "2014-11-03" "2014-11-03" ...
##  $ TradeDateTime   : POSIXct, format: "2014-11-03 10:00:00" "2014-11-03 11:00:00" ...
##  $ n.trades        : int  1607 2055 3417 3686 3978 4707 5168 250 1602 1203 ...
##  $ last.price      : num  16.1 16.1 16.2 16.1 16.1 ...
##  $ weighted.price  : num  16.1 16.1 16.2 16.2 16.1 ...
##  $ period.ret      : num  -0.00864 0.00124 0.0056 -0.00124 -0.00372 ...
##  $ period.ret.volat: num  0.000325 0.000324 0.000278 0.000235 0.000263 ...
##  $ sum.qtd         : num  824900 926700 1408500 1034900 1141100 ...
##  $ sum.vol         : num  13291157 14907444 22757436 16729199 18362060 ...
##  $ n.buys          : int  579 1113 1888 2265 1972 1878 2309 23 659 526 ...
##  $ n.sells         : int  1028 942 1529 1421 2006 2829 2859 227 943 677 ...
##  $ Tradetime       : chr  "10:00:00" "11:00:00" "12:00:00" "13:00:00" ...
\end{verbatim}

Agora irei criar um banco de dados para cada ação e depois obter os log
retornos.

\begin{Shaded}
\begin{Highlighting}[]
\KeywordTok{library}\NormalTok{(dplyr)}
\NormalTok{dados_ITSA4 <-}\StringTok{ }\KeywordTok{filter}\NormalTok{(dados_top6, InstrumentSymbol ==}\StringTok{ "ITSA4"}\NormalTok{) %>%}\StringTok{ }
\StringTok{    }\KeywordTok{select}\NormalTok{(SessionDate, weighted.price) %>%}\StringTok{ }\KeywordTok{mutate}\NormalTok{(}\DataTypeTok{log_retorno =} \KeywordTok{log}\NormalTok{(weighted.price) -}\StringTok{ }
\StringTok{    }\KeywordTok{lag}\NormalTok{(}\KeywordTok{log}\NormalTok{(weighted.price)))}
\NormalTok{dados_PETR4 <-}\StringTok{ }\KeywordTok{filter}\NormalTok{(dados_top6, InstrumentSymbol ==}\StringTok{ "PETR4"}\NormalTok{) %>%}\StringTok{ }
\StringTok{    }\KeywordTok{select}\NormalTok{(SessionDate, weighted.price) %>%}\StringTok{ }\KeywordTok{mutate}\NormalTok{(}\DataTypeTok{log_retorno =} \KeywordTok{log}\NormalTok{(weighted.price) -}\StringTok{ }
\StringTok{    }\KeywordTok{lag}\NormalTok{(}\KeywordTok{log}\NormalTok{(weighted.price)))}
\NormalTok{dados_ITUB4 <-}\StringTok{ }\KeywordTok{filter}\NormalTok{(dados_top6, InstrumentSymbol ==}\StringTok{ "ITUB4"}\NormalTok{) %>%}\StringTok{ }
\StringTok{    }\KeywordTok{select}\NormalTok{(SessionDate, weighted.price) %>%}\StringTok{ }\KeywordTok{mutate}\NormalTok{(}\DataTypeTok{log_retorno =} \KeywordTok{log}\NormalTok{(weighted.price) -}\StringTok{ }
\StringTok{    }\KeywordTok{lag}\NormalTok{(}\KeywordTok{log}\NormalTok{(weighted.price)))}
\NormalTok{dados_BBDC4 <-}\StringTok{ }\KeywordTok{filter}\NormalTok{(dados_top6, InstrumentSymbol ==}\StringTok{ "BBDC4"}\NormalTok{) %>%}\StringTok{ }
\StringTok{    }\KeywordTok{select}\NormalTok{(SessionDate, weighted.price) %>%}\StringTok{ }\KeywordTok{mutate}\NormalTok{(}\DataTypeTok{log_retorno =} \KeywordTok{log}\NormalTok{(weighted.price) -}\StringTok{ }
\StringTok{    }\KeywordTok{lag}\NormalTok{(}\KeywordTok{log}\NormalTok{(weighted.price)))}
\NormalTok{dados_ABEV3 <-}\StringTok{ }\KeywordTok{filter}\NormalTok{(dados_top6, InstrumentSymbol ==}\StringTok{ "ABEV3"}\NormalTok{) %>%}\StringTok{ }
\StringTok{    }\KeywordTok{select}\NormalTok{(SessionDate, weighted.price) %>%}\StringTok{ }\KeywordTok{mutate}\NormalTok{(}\DataTypeTok{log_retorno =} \KeywordTok{log}\NormalTok{(weighted.price) -}\StringTok{ }
\StringTok{    }\KeywordTok{lag}\NormalTok{(}\KeywordTok{log}\NormalTok{(weighted.price))) %>%}\StringTok{ }\KeywordTok{mutate}\NormalTok{(}\DataTypeTok{log_retorno =} \KeywordTok{log}\NormalTok{(weighted.price) -}\StringTok{ }
\StringTok{    }\KeywordTok{lag}\NormalTok{(}\KeywordTok{log}\NormalTok{(weighted.price)))}
\NormalTok{dados_BBSE3 <-}\StringTok{ }\KeywordTok{filter}\NormalTok{(dados_top6, InstrumentSymbol ==}\StringTok{ "BBSE3"}\NormalTok{) %>%}\StringTok{ }
\StringTok{    }\KeywordTok{select}\NormalTok{(SessionDate, weighted.price) %>%}\StringTok{ }\KeywordTok{mutate}\NormalTok{(}\DataTypeTok{log_retorno =} \KeywordTok{log}\NormalTok{(weighted.price) -}\StringTok{ }
\StringTok{    }\KeywordTok{lag}\NormalTok{(}\KeywordTok{log}\NormalTok{(weighted.price)))}
\end{Highlighting}
\end{Shaded}

Removendo \texttt{NA}s.

\begin{Shaded}
\begin{Highlighting}[]
\NormalTok{dados_BBSE3 <-}\StringTok{ }\NormalTok{dados_BBSE3[}\DecValTok{2}\NormalTok{:}\DecValTok{3778}\NormalTok{, ]}
\NormalTok{dados_ABEV3 <-}\StringTok{ }\NormalTok{dados_ABEV3[}\DecValTok{2}\NormalTok{:}\DecValTok{3778}\NormalTok{, ]}
\NormalTok{dados_BBDC4 <-}\StringTok{ }\NormalTok{dados_BBDC4[}\DecValTok{2}\NormalTok{:}\DecValTok{3778}\NormalTok{, ]}
\NormalTok{dados_ITUB4 <-}\StringTok{ }\NormalTok{dados_ITUB4[}\DecValTok{2}\NormalTok{:}\DecValTok{3778}\NormalTok{, ]}
\NormalTok{dados_PETR4 <-}\StringTok{ }\NormalTok{dados_PETR4[}\DecValTok{2}\NormalTok{:}\DecValTok{3777}\NormalTok{, ]}
\NormalTok{dados_ITSA4 <-}\StringTok{ }\NormalTok{dados_ITSA4[}\DecValTok{2}\NormalTok{:}\DecValTok{3778}\NormalTok{, ]}
\end{Highlighting}
\end{Shaded}

Criando matriz com os dados diários.

\begin{Shaded}
\begin{Highlighting}[]
\NormalTok{matriz_logrtn <-}\StringTok{ }\KeywordTok{data.frame}\NormalTok{(}\DataTypeTok{lrtn_ITSA4 =} \NormalTok{dados_ITSA4$log_retorno, }\DataTypeTok{lrtn_ITUB4 =} \NormalTok{dados_ITUB4$log_retorno, }
    \DataTypeTok{lrtn_BBDC4 =} \NormalTok{dados_BBDC4$log_retorno, }\DataTypeTok{lrtn_ABEV3 =} \NormalTok{dados_ABEV3$log_retorno, }
    \DataTypeTok{lrtn_BBSE3 =} \NormalTok{dados_BBSE3$log_retorno)}
\KeywordTok{summary}\NormalTok{(matriz_logrtn)}
\end{Highlighting}
\end{Shaded}

\begin{verbatim}
##    lrtn_ITSA4           lrtn_ITUB4           lrtn_BBDC4        
##  Min.   :-1.079e-01   Min.   :-9.911e-02   Min.   :-1.976e-01  
##  1st Qu.:-3.412e-03   1st Qu.:-3.276e-03   1st Qu.:-3.698e-03  
##  Median :-1.467e-04   Median :-5.178e-05   Median :-3.910e-06  
##  Mean   :-4.198e-05   Mean   :-5.640e-06   Mean   :-5.718e-05  
##  3rd Qu.: 3.154e-03   3rd Qu.: 3.205e-03   3rd Qu.: 3.599e-03  
##  Max.   : 7.451e-02   Max.   : 8.019e-02   Max.   : 7.990e-02  
##    lrtn_ABEV3           lrtn_BBSE3        
##  Min.   :-4.289e-02   Min.   :-6.050e-02  
##  1st Qu.:-2.287e-03   1st Qu.:-3.531e-03  
##  Median : 7.486e-05   Median : 1.017e-05  
##  Mean   : 5.483e-05   Mean   :-2.386e-05  
##  3rd Qu.: 2.359e-03   3rd Qu.: 3.362e-03  
##  Max.   : 2.241e-02   Max.   : 8.743e-02
\end{verbatim}

\begin{Shaded}
\begin{Highlighting}[]
\KeywordTok{cor}\NormalTok{(matriz_logrtn)}
\end{Highlighting}
\end{Shaded}

\begin{verbatim}
##            lrtn_ITSA4 lrtn_ITUB4 lrtn_BBDC4 lrtn_ABEV3 lrtn_BBSE3
## lrtn_ITSA4  1.0000000  0.8441257  0.7255955  0.4612668  0.5651589
## lrtn_ITUB4  0.8441257  1.0000000  0.7891058  0.4916024  0.5817463
## lrtn_BBDC4  0.7255955  0.7891058  1.0000000  0.4387216  0.5068598
## lrtn_ABEV3  0.4612668  0.4916024  0.4387216  1.0000000  0.4038272
## lrtn_BBSE3  0.5651589  0.5817463  0.5068598  0.4038272  1.0000000
\end{verbatim}

\begin{Shaded}
\begin{Highlighting}[]
\KeywordTok{library}\NormalTok{(MTS)}
\KeywordTok{MTSplot}\NormalTok{(matriz_logrtn)}
\end{Highlighting}
\end{Shaded}

\includegraphics{Baixando_dados_diarios_files/figure-latex/unnamed-chunk-6-1.pdf}

\begin{Shaded}
\begin{Highlighting}[]
\KeywordTok{head}\NormalTok{(matriz_logrtn)}
\end{Highlighting}
\end{Shaded}

\begin{verbatim}
##      lrtn_ITSA4    lrtn_ITUB4   lrtn_BBDC4    lrtn_ABEV3   lrtn_BBSE3
## 1 -0.0079058649 -0.0057653310 -0.004521064 -0.0016150278 -0.010986101
## 2  0.0050112343  0.0045462710 -0.002177734  0.0043904510  0.008955299
## 3 -0.0030588852 -0.0059963761 -0.001079848  0.0004751700 -0.004649955
## 4 -0.0036985681 -0.0022077264 -0.006330370 -0.0045601580 -0.009555026
## 5  0.0019715511  0.0028860314  0.003117587 -0.0008106029  0.013210150
## 6  0.0002158613 -0.0003814315 -0.004567171 -0.0029404438 -0.006192878
\end{verbatim}

\begin{Shaded}
\begin{Highlighting}[]
\NormalTok{ccm}
\end{Highlighting}
\end{Shaded}

\begin{verbatim}
## function (x, lags = 12, level = FALSE, output = T) 
## {
##     if (!is.matrix(x)) 
##         x = as.matrix(x)
##     nT = dim(x)[1]
##     k = dim(x)[2]
##     if (lags < 1) 
##         lags = 1
##     y = scale(x, center = TRUE, scale = FALSE)
##     V1 = cov(y)
##     if (output) {
##         print("Covariance matrix:")
##         print(V1, digits = 3)
##     }
##     se = sqrt(diag(V1))
##     SD = diag(1/se)
##     S0 = SD %*% V1 %*% SD
##     ksq = k * k
##     wk = matrix(0, ksq, (lags + 1))
##     wk[, 1] = c(S0)
##     j = 0
##     if (output) {
##         cat("CCM at lag: ", j, "\n")
##         print(S0, digits = 3)
##         cat("Simplified matrix:", "\n")
##     }
##     y = y %*% SD
##     crit = 2/sqrt(nT)
##     for (j in 1:lags) {
##         y1 = y[1:(nT - j), ]
##         y2 = y[(j + 1):nT, ]
##         Sj = t(y2) %*% y1/nT
##         Smtx = matrix(".", k, k)
##         for (ii in 1:k) {
##             for (jj in 1:k) {
##                 if (Sj[ii, jj] > crit) 
##                   Smtx[ii, jj] = "+"
##                 if (Sj[ii, jj] < -crit) 
##                   Smtx[ii, jj] = "-"
##             }
##         }
##         if (output) {
##             cat("CCM at lag: ", j, "\n")
##             for (ii in 1:k) {
##                 cat(Smtx[ii, ], "\n")
##             }
##             if (level) {
##                 cat("Correlations:", "\n")
##                 print(Sj, digits = 3)
##             }
##         }
##         wk[, (j + 1)] = c(Sj)
##     }
##     if (output) {
##         par(mfcol = c(k, k))
##         k0 = 4
##         if (k > k0) 
##             par(mfcol = c(k0, k0))
##         tdx = c(0, 1:lags)
##         jcnt = 0
##         if (k > 10) {
##             print("Skip the plots due to high dimension!")
##         }
##         else {
##             for (j in 1:ksq) {
##                 plot(tdx, wk[j, ], type = "h", xlab = "lag", 
##                   ylab = "ccf", ylim = c(-1, 1))
##                 abline(h = c(0))
##                 crit = 2/sqrt(nT)
##                 abline(h = c(crit), lty = 2)
##                 abline(h = c(-crit), lty = 2)
##                 jcnt = jcnt + 1
##                 if ((jcnt == k0^2) && (k > k0)) {
##                   jcnt = 0
##                   cat("Hit Enter for more plots:", "\n")
##                   readline()
##                 }
##             }
##         }
##         par(mfcol = c(1, 1))
##         cat("Hit Enter for p-value plot of individual ccm: ", 
##             "\n")
##         readline()
##     }
##     r0i = solve(S0)
##     R0 = kronecker(r0i, r0i)
##     pv = rep(0, lags)
##     for (i in 1:lags) {
##         tmp = matrix(wk[, (i + 1)], ksq, 1)
##         tmp1 = R0 %*% tmp
##         ci = crossprod(tmp, tmp1) * nT * nT/(nT - i)
##         pv[i] = 1 - pchisq(ci, ksq)
##     }
##     if (output) {
##         plot(pv, xlab = "lag", ylab = "p-value", ylim = c(0, 
##             1))
##         abline(h = c(0))
##         abline(h = c(0.05), col = "blue")
##         title(main = "Significance plot of CCM")
##     }
##     ccm <- list(ccm = wk, pvalue = pv)
## }
## <environment: namespace:MTS>
\end{verbatim}

\begin{Shaded}
\begin{Highlighting}[]
\KeywordTok{mq}\NormalTok{(matriz_logrtn)}
\end{Highlighting}
\end{Shaded}

\begin{verbatim}
## Ljung-Box Statistics:  
##         m       Q(m)     df    p-value
##  [1,]     1       156      25        0
##  [2,]     2       202      50        0
##  [3,]     3       252      75        0
##  [4,]     4       265     100        0
##  [5,]     5       298     125        0
##  [6,]     6       318     150        0
##  [7,]     7       349     175        0
##  [8,]     8       408     200        0
##  [9,]     9       442     225        0
## [10,]    10       465     250        0
## [11,]    11       484     275        0
## [12,]    12       513     300        0
## [13,]    13       533     325        0
## [14,]    14       554     350        0
## [15,]    15       583     375        0
## [16,]    16       651     400        0
## [17,]    17       682     425        0
## [18,]    18       713     450        0
## [19,]    19       744     475        0
## [20,]    20       764     500        0
## [21,]    21       783     525        0
## [22,]    22       811     550        0
## [23,]    23       837     575        0
## [24,]    24       864     600        0
\end{verbatim}

\includegraphics{Baixando_dados_diarios_files/figure-latex/unnamed-chunk-9-1.pdf}

\hypertarget{refs}{}

\end{document}
